\documentclass[oneside,a4paper]{book}
%\pagestyle{headings}
\frontmatter
\input{preamble}

% A B S T R A C T
% % % % % % % % % % % % % % % % % % % % % % % % % % % % % % % % % %
\chapter*{\centering Abstract}
\begin{quotation}
\noindent

\textbf{Not an \emph{abstract} yet, but the original project description:}

Ziel des Projekts ist, ein empirisch abgest�tztes Instrument f�r den Programmier-Unterricht am Gymnasium zu entwickeln, in welchem Sch�ler:innen verschiedene Abstraktionsebenen interaktiv erleben k�nnen.

Auf der Basis von Processing mit Python Syntax (https://py.processing.org/) soll der einerseits der visuelle Ablauf eines Programms, aber auch die Parsing-Schritte und die �bersetzung in Byte-Code Seite-an-Seite sicht- und untersuchbar gemacht werden, damit Sch�ler:innen die Auswirkungen ihres Programmcodes auf die Maschine live erleben k�nnen.

Die Entwicklung des Produkts wird theoretisch begleitet und das Produkt selbst empirisch gepr�ft werden.

Als Basis der Umsetzung dient Glamorous Toolkit, eine Entwicklungsumgebung basierend auf Smalltalk/Pharo, welche u.a. von Oscar Nierstrasz f�r Master- und Doktoratsstudieng�nge weiterentwickelt worden ist.

\end{quotation}
\clearpage


% C O N T E N T S
% % % % % % % % % % % % % % % % % % % % % % % % % % % % % % % % % % % % % % % %
\tableofcontents

\mainmatter
%%%%% Introduction %%%%%
\chapter{Introduction}

%%%%% The Problem %%%%%
\chapter{Teaching Programming}
\section{Didactic Approaches}
See e.g. \cite{Sch11} and \cite{Mod16}.
\section{Personal Experience}

%%%%% Related Work %%%%%
\chapter{Programming Environments}
\section{Integrated Development Environment (IDE)}
Such as VS Code
\section{Didactic Programming Environments}
Such as Thonny
\section{Low-level Programming}
With "Little Man Computer" or "Human Resource Machine"

%%%%% The Solution %%%%%
\chapter {Processing Abstractions}

%%%%% The Validation %%%%%
\chapter{PA in Practice}

%%%%% Conclusion and Future Work %%%%%
\chapter{Conclusion}


%END Doc
%-------------------------------------------------------

\pagebreak % References %%%%%%%%%%%%%%%%%%%%%%%%%%%%%%%%%%%%%%%%%%

% \bibliography{thesis}
\bibliographystyle{plain}

\begin{raggedright}\bibliography{thesis}\end{raggedright}
\addcontentsline{toc}{chapter}{References}

\end{document}
