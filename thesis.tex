\documentclass[oneside,a4paper]{book}
%\pagestyle{headings}
\frontmatter
\input{preamble}

% A B S T R A C T
% % % % % % % % % % % % % % % % % % % % % % % % % % % % % % % % % %
\chapter*{\centering Abstract}
\begin{quotation}
\noindent

\textbf{Not an \emph{abstract} yet, but the original project description:}

Ziel des Projekts ist, ein empirisch abgest�tztes Instrument f�r den Programmier-Unterricht am Gymnasium zu entwickeln, in welchem Sch�ler:innen verschiedene Abstraktionsebenen interaktiv erleben k�nnen.

Auf der Basis von Processing mit Python Syntax (https://py.processing.org/) soll der einerseits der visuelle Ablauf eines Programms, aber auch die Parsing-Schritte und die �bersetzung in Byte-Code Seite-an-Seite sicht- und untersuchbar gemacht werden, damit Sch�ler:innen die Auswirkungen ihres Programmcodes auf die Maschine live erleben k�nnen.

Die Entwicklung des Produkts wird theoretisch begleitet und das Produkt selbst empirisch gepr�ft werden.

Als Basis der Umsetzung dient Glamorous Toolkit, eine Entwicklungsumgebung basierend auf Smalltalk/Pharo, welche u.\,a. von Oscar Nierstrasz f�r Master- und Doktoratsstudieng�nge weiterentwickelt worden ist.

\end{quotation}
\clearpage


% C O N T E N T S
% % % % % % % % % % % % % % % % % % % % % % % % % % % % % % % % % % % % % % % %
\tableofcontents

\mainmatter
%%%%% Introduction %%%%%
\chapter{Introduction}
Programming with Processing vs. "Little Man Computer" or "Human Resource Machine"

%%MAIN: thesis.tex

%%%%% The Problem %%%%%
%%%%% Related Work %%%%%
\chapter{Leaky Abstractions when Teaching Programming} \label{ch_theory}
% On the lack of connecting high-level languages with lower-level concepts

Before introducing the product of this thesis in chapter \ref{ch_pa}, we first introduce the problem it should help solve: How abstractions involved in programming are taught.

In detail, we first introduce the concept of (leaky) abstractions in multitier architectures in section \ref{sc_abstractions}; show how these are discussed in didactic literature in sections \ref{sc_didactic}; and how common IDEs already support handling these difficulties (in section \ref{sc_ides}).


\section{Multitier Architectures and (Leaky) Abstractions} \label{sc_abstractions}
In order to handle complexities arising in both theoretical and practical computer science, subjects are split into multiple layers or tiers to be described, investigated and used separatedly. \xxx{citation needed?}

Common such multitier architectures taught at high school level are the networking stack (either the seven layered OSI model or the simplified four layered DoD architecture) or the software-hardware stack ranging from apps and hardware abstracting OS down to transistors consisting of e.\,g. silicium atoms.

\xxx{diagram of such an architecture?}

Ideally, in such architectures all layers above the layer to be investigated can be ignored (beyond what the layer will be used for) and all the layers below can be abstracted away into a nicely defined interface.

As such, programming should be possible to be done independently of hardware and even the operating system, in the same way that natural languages can be taught independently of body or mind of the students.

In his article ''The Law of Leaky Abstractions'' \citep{Spo02} introduces the concept of \emph{leaky abstractions}, claiming that for all non-trivial such architectures, details of lower layers are to some degree bound to bleed through to upper layers. In other words, in practice complex interfaces tend to be incomplete or 'leaky'.

In teaching computer science, such leaky abstractions occur repeatedly, e.\,g. when an app doesn't run on a different device (with either the OS or the processor architecture leaking); or when a document seemingly can't be saved (with either the file system or differences between apps leaking).

More specifically, in programming there are several ways of abstracting away technical details:

\begin{itemize}
\item Programming instructions consist of source code which consists of encoded bits which are stored in memory or on a drive.
\item Source code consists of tokens which are usually parsed into an abstract syntax tree (AST) which are either directly or via intermediary representations translated into machine code to be run on a virtual or actual machine.
\item When programming instructions through the above abstractions are executed, variable values are encoded and stored in memory, function calls are tracked through a call stack, input state is continually mapped into memory and output is generated in several forms -- where e.g. textual output causes a font renderer to interpret glyph instructions for every character; or graphical output is anti-aliased before any pixel data is produced.
\end{itemize}
\xxx{citation needed?}

Of these different layers, students usually focus on turining instructions into source code and then checking the program's output -- or any error messages produced by the compiler or interpreter (see section \ref{sc_didactic}). Still, several of the lower layered abstractions might leak through, such as:

\begin{itemize}
\item Missing a stop condition in a recursive function leads to a cryptic ''Stack overflow'' error -- leaking information about the call stack.
\item If a program outputs emojis, they might look notably differently in source code and output -- leaking font rendering.
\item Similarly, programs containing emojis might have emojis garbled depending on the app used for inspecting the source code -- leaking text encoding.
\item If a program contains an endless loop, there might be neither error message nor output, so that it might wrongly seem that the computer isn't doing anything. This isn't an abstraction leak in the above sense but a related student misconception.
\end{itemize}


\section{Didactic Approaches} \label{sc_didactic}
See e.\,g. \cite{Sch11}, \cite{Mod16}, \cite{Har20} or \cite{Lee20} only focusing on one aspect

\subsection{Teaching Top Down}
Working downwards from gaming, as in \cite{Wei16}

\subsection{Teaching Bottom Up}
Running Tetris on NANDs as described in \cite{Cak17}, \cite{Nis21}


\section{Abstractions in IDEs} \label{sc_ides}
% Brief overview over views offered by common IDEs (such as VS Code) but mainly didactic ones such as Thonny (\cite{Ann15}), Mu (\cite{Tol23}), \emph{etc.}.

Integrated development environments used for programming offer a variety of different views on a program beyond its source code and its runtime output. The popular Visual Studio Code offers e.\,g. through extensions step-by-step debugging with variables and the call stack listed \cite{Mic25}. This is mirrored in most other full fledged IDEs such as PyCharm \citep{Jet25} or Eclipse \citep{Ecl25}.

And while such IDEs through appropriate extensions even allow inspecting Python bytecode, the respective views are usually overwhelming for programming novices and thus rather targetted at professional developers than high school students.

As a remedy, several teaching oriented IDEs have been developped, such as ''Code with Mu'' which offers a minimal command set and still allows runtime inspection \citep{Tol23}; or Thonny which had the goal to visualize runtime concepts beyond what IDEs offered at the time \citep[p. 119]{Ann15}:

On the one hand, Thonny shows intermediary steps during expression evaluation. This demonstrates that statements are not evaluated in one go, but indeed in a predetermined order operation by operation.\footnote{In professional IDEs, intermediary results are usually available by hovering over a specific operator with the order of evaluation being left to the user to determine.}

On the other hand, Thonny visualizes recursion by showing code in a new pop-up for every function call, so that multiple recursive function calls lead to an equivalent number of visible pop-ups. Most other IDEs rather show a call stack as in a separate view, which abstracts the stack into a list.\footnote{As a compromise, Glamorous Toolkit presented in chapter \ref{ch_background} displays the call stack as a list of expandable method sources with the call location highlighted.}

Finally, Thonny distinguishes between values on the stack and on the heap, showing the pointer to the heap as the value actually pushed on the stack and in a separate view the actual object on the heap at the given address.

Thus, the Thonny IDE set out to and indeed nicely visualizes several concepts on lower runtime layers.

\cite{Jal22} has assembled a list of tools targetted at visualizing some of these concepts outside of an IDE. One noteable such alternative approach is taken by Python Tutor \citep{Pyt25} which combines a visualization of stack frames variable values as pointers and deconstructed objects.

%%%%% Background Information %%%%%

\chapter{Technical Background} \label{ch_background}

Before delving into this thesis' product, an overview of the technologies involved is given in this chapter: The environment described is implemented on the Glamorous Toolkit platform, which will be introduced in \ref{sc_gt}, using Moldable Development patterns (a concept introduced in \ref{sc_moldable}). Finally, as a teaching language, we have chosen Processing for which an overview is given in \ref{sc_processing}.



\section{Processing} \label{sc_processing}

Processing is a programming language consisting of a graphics \ac{API} built upon a mainstream language as a base. Development started between 1997 and 2004 at the MIT Media Lab as a continuation of the Design By Numbers project with the goal of creating a unified environment for teaching art students the fundamentals of programming as a basis for creating digital, visual art.

While the original Design By Numbers integrated the language into an \ac{IDE}, having input and output side by side, it used its own, simplified programming language \cite{DBN01}. Processing's authors, Reas and Fry, based their language upon then popular and portable Java, removing much of the boilerplate required for object orientation, enhancing it with visual primitives and implicitly showing an output window, allowing for quick results (see figure \ref{fig_alpinerWanderweg}).

\begin{cfigure}[fig_alpinerWanderweg]{Example code (with Java syntax) and output}
\begin{minipage}{.5\textwidth}
\begin{code}
// Output canvas dimensions
size(200, 200);
// (Default white) square
rect(50, 50, 100, 100);
// Red inner rectangle
fill(255, 0, 0);
rect(50, 50 + 100 / 3, 100, 100 / 3);
\end{code}
\end{minipage}
\begin{minipage}{.45\textwidth}
\centering
\includegraphics[height=3cm]{alpinerWanderweg}
\end{minipage}
\end{cfigure}

Inside the \ac{IDE}, Processing code is compiled to Java bytecode and run inside the same \ac{JVM} as the \ac{IDE}. The Processing \ac{API} was thus provided in the form of compiled Java code, and this hasn't changed for the official Processing \ac{IDE} to this day.

Apart from graphical primitives, Processing features an implicit event loop, which allows for creating (interactive) animations within a dozen lines of code (see figure \ref{fig_jumpingBall}). Reacting to input happens either by polling while painting a frame (for this there are implicit global variables such as \ct{mousePressed}, \ct{keyCode}, \etc) or by defining event handlers alongside \ct{setup} and \ct{draw} (such as \ct{mouseClicked(event)} or \ct{keyTyped(event)}).

\begin{cfigure}[fig_jumpingBall]{Example code (with Python syntax) and four output frames}
\begin{minipage}{.5\textwidth}
\begin{code}
y = 50; dy = 0

# called once after global code
def setup():
    size(100, 200)

# called repeatedly for every frame
def draw():
    global y, dy
    background(192)
    circle(50, y, 50)
    y += dy; dy += 1
    if y > height - 25:
        dy = -0.9 * dy
\end{code}
\end{minipage}
\begin{minipage}{.45\textwidth}
\centering
\includegraphics[height=3cm]{ball1}
\includegraphics[height=3cm]{ball2}
\includegraphics[height=3cm]{ball3}
\includegraphics[height=3cm]{ball4}
\end{minipage}
\end{cfigure}

Since Python has become the prevalent teaching language \cite{Cod20}, Processing has been extended with a Python mode, which uses Python as a basis, with the Processing \ac{API} being available by default and the animation loop still being implicit \cite{Pro25}.

As the official Processing \ac{IDE} remains implemented in Java, Processing's official Python mode uses the Jython library to compile the code to Java Bytecode, so that it can be run in the same way as Processing programs written in the original Java mode \cite{Jyt25}. This also gives access to most of Python's vast standard library, with the exception of a few modules that were precompiled to native code for the various platforms for speed reasons and thus had to be rewritten for or left out of Jython.

In fact, as the \ac{JVM} is sufficiently generic to be the target for a wide variety of other languages, further modes for JavaScript\footnote{Based on the Rhino compiler from \archivedurl{https://rhino.github.io/}.} or R\footnote{Based on the Renjin interpreter from \archivedurl{https://renjin.org/}.} have been added. This allows Processing and its dedicated \ac{IDE} to be used as a starting point for programming and later seamlessly transitioning to the desired language, such as pure Python, which remains part of the motivation for students: learning an ``actually useful'' language.

Since developers have started moving away from the \ac{JVM}, there are now several reimplementations of Processing, such as p5.js for running Processing on top of JavaScript in a web environment,\footnote{\emph{Cf.} \archivedurl{https://p5js.org/} and try it out at \url{https://editor.p5js.org/}.} p5.py for running Processing in a pure Python environment,\footnote{Requiring two additional lines: \ct{from p5 import *} at the top and \ct{run()} at the bottom; \emph{cf.} \url{https://github.com/gromko/p5-python}.} or a version of Processing for microcontrollers such as Arduino.\footnote{\emph{Cf.} \archivedurl{https://www.arduino.cc/education/visualization-with-arduino-and-processing/}.} With this thesis, a limited version for a Smalltalk environment is also available (see chapter \ref{ch_pa} and appendix \ref{app_api} for an \ac{API} overview).

In fact, when we started teaching programming in high school classes, we initially ran our own \ac{IDE} based on web technologies and p5.js with custom error handling and support for live programming\footnote{This \ac{IDE} is still available at \archivedurl{https://software.zeniko.ch/ProcessingIDE.zip}. Note that it is targeted at \ct{mshta.exe} and, as such, runs best under Microsoft Windows.} before changing to the official \ac{IDE} for its Python mode. Our experience of working with Processing with ninth and tenth graders over the past decade has shown that it allows novice programmers to learn enough of the language within a month that they are able to write a clone of a game like Pong, Flappy Bird or Geometry Dash as a group project. Feedback from the various student groups about this part of the computer science curriculum has always been positive to very positive (and remains so, as we will see in chapter \ref{ch_practice}).

Reasons to use Processing are thus manifold: Processing allows a top-down approach starting with visual art, which allows teachers to motivate students with less interest in mathematics and natural sciences. Furthermore, it quickly yields pleasant-looking results, which also adds to the initial motivation \cite{Chi23}. Additionally, Processing has a large community sharing sketches and ideas, which can be used as inspiration for both students and teachers. Then, it can be based on currently widely used languages such as Python, which allows using it as a stepping stone and makes it a ``real'' programming language in the eyes of novices. In contrast, Processing itself is sufficiently unknown that even students already experienced with programming will have something new to discover. Finally, it has proven itself in our own experience over the years.



\section{Moldable Development} \label{sc_moldable}

Moldable development is a term coined by Chi\c{s}, Nierstrasz and G�rba \cite{Chi15,Chi16,Gir22} for a software development approach that should make it easier to understand a computer system by extending (``molding'') it with views and features. The goal of moldable development is to quickly get feedback on code and objects being worked on, so that a programmer can confidently make appropriate changes.

In traditional \acp{IDE}, a running system is inspected either through its source code or its live runtime objects. Available views (see \ref{ssc_ides}) are static, and new views are added through non-trivial extensions. Moldable development works in an environment in which a tool is more easily adaptable to data, making it simple to write either one-off throw-away views and tools, but also allowing developers to refactor such throw-away code into reusable components when needed.

Moldable development is thus a form of exploratory programming (cf. \ref{ssc_exploration}) on live objects where tools, whether one-off or reusable, are created in a bottom-up approach with immediate feedback available at every step. See \eg figure \ref{fig_moldable_screenshot} where the compilation of a Processing program is explored for what information about the produced bytecode to show with the goal of having one-off code sufficiently generalized that it can be added as a reusable view for all objects of this type (as eventually seen in figure \ref{fig_gt_screenshot}). Such exploration code could also later be extracted into tests, ensuring that what worked once will continue to work.

\begin{cfigure}[fig_moldable_screenshot]{The moldable \acs{GT} environment with data structures being explored for creating a view of an aspect (here: bytecode for a Smalltalk method)}
\includegraphics[width=.7\textwidth]{moldable_screenshot}
\end{cfigure}

In order to support this, a moldable environment must have extensibility in its core, allowing tools and views to be registered \eg through a simple code annotation of a few characters, which the environment can use to detect and include them (instead of having to write a lot of configuration boilerplate and overhead, which \ac{IDE} extensions meant for independent distribution usually involve).

Nierstrasz and G�rba \cite{Nie24} identified several development patterns that are common to or supportive of moldable development. One core pattern of moldable development is the ``Moldable Object'': Objects should be implementable incrementally with live object states and previously developed views remaining available throughout the whole process. An object initially consisting of little more than a data wrapper is thus extended with new functionality as it fits the available live data -- instead of designing an object on a clean slate or along tests. Extending objects iteratively based on actual needs should also ensure that code is cleanly separated, as during the exploration phase, it should become clear where code fits best.

Having a moldable environment also allows for working on code and documentation intertwinedly, similar to literate programming \cite{Knu84}. In contrast to literate programming, where code has to be extracted first, in moldable development, every code snippet should be runnable on its own, and besides code and documentation, also live results can be included. This allows a moldable environment to be used to either first document ideas and then add matching code but also to document progress or explain written code (which can then easily be extracted into a test case).

For students, such a Project Diary pattern could be used as a learning journal (similar to Microsoft OneNote\footnote{\emph{Cf.} \archivedurl{https://www.microsoft.com/de-ch/microsoft-365/onenote/digital-note-taking-app}.}), for project exploration (similar to Jupyter notebooks\footnote{\emph{Cf.} \archivedurl{https://docs.jupyter.org/}.}), or for project documentation. Another useful pattern for teaching is the ``Composed Narrative'' that visualizes object relations through side-by-side views tailored towards explaining a relation or interaction.



\section{Glamorous Toolkit} \label{sc_gt}

\acf{GT} is a fully programmable environment optimized for moldable development (see \ref{sc_moldable}), consisting of a Smalltalk \ac{VM} and runtime environment, a custom user interface, and the source code of the Smalltalk code for everything running within it. By default, it persists its entire state into a system image, so that live objects don't have to be recreated at restart and may outlive their original source \cite{Gir23}.


\subsection{Smalltalk VM}

Smalltalk is a fully object-oriented language based on message passing\footnote{One of many characteristics that Smalltalk shares with Java.} that was originally designed for educational use and, as a consequence, has rather minimalist syntax that is supposed to read more naturally: its syntax limits the need for parentheses, aligns punctuation with natural language (using full stops to end a statement and semicolons to continue a statement by sending another message to the same object) and interweaves a message's name with its arguments\footnote{\emph{Cf.} \eg the \ct{#ifTrue:ifFalse} message in figure \ref{fig_annotated_view} where each argument follows part of the name. Note that these are not the argument's names; those are declared separately in the definition of \ct{#ifTrue:ifFalse}.} \cite{Gol83}.

One potential issue for programmers experienced in ALGOL-68-derived languages is operator precedence, which in Smalltalk is limited for simplicity to just three different levels: (1) messages without arguments; (2) binary operators (which in contrast to mathematics and most other languages discussed in this thesis all have the same precedence and left associativity, and are of course also implemented as messages); and (3) all other messages.

At \ac{GT}'s core is the OpenSmalltalk Cog \ac{VM}.\footnote{\emph{Cf.} \url{https://github.com/OpenSmalltalk/opensmalltalk-vm}.} The Cog \ac{VM} is open source (MIT licensed) and shared with other Smalltalk based environments, in particular \ac{GT}'s predecessors (see \ref{ssc_gt_history}). Its source code is written in a subset of the Smalltalk language \cite{Ing97}, which is transpiled to C both for performance and for achieving cross-platform compatibility by relying on the various available C compilers. As a consequence, \ac{GT} runs on Unix systems just as well as under Microsoft Windows.

Smalltalk and the Cog \ac{VM} are highly reflective, allowing access to all but the most fundamental built-ins. In fact, all messages passed are primarily implemented in Smalltalk, but common operations can be forwarded to native code with a \ct{<primitive:...>} pragma annotation, with a fallback being provided in Smalltalk in case the native implementation fails. In particular, the execution context and the compiled bytecode of any message are available for inspection and modification. This allows users to customize the environment entirely to their liking.

For performance, the Cog \ac{VM} includes a \ac{JIT} for compiling methods called multiple times to native code on the fly \cite{Ope25}. More recently, for further optimizations, an adaptive optimizer named ``Speculative Inlining Smalltalk Architecture'' (SISTA) has been introduced by Cl�ment B�ra \cite{Ber17}, which also enables saving the optimized methods into the image, thus persisting them between restarts of the \ac{VM}. The bytecodes used for \ac{GT} are thus those proposed by B�ra and Miranda \cite{Ber14} and diverge to some extent from the original Smalltalk-80 bytecode format \cite[p.\,596]{Gol83}, in particular by enabling (more) multi-byte instructions that allow compilers to inline more common objects and code.

With \ac{GT} being based on a Smalltalk \ac{VM}, Smalltalk is \ac{GT}'s primary language. Support for other popular languages such as Python, JavaScript, or Java is, however, possible by connecting to an external runtime through the \ct{LanguageLink} protocol, \ie by passing serialized objects over sockets \cite{Fra24}.\footnote{The serialization format chosen is either JSON or the more compact binary representation MessagePack (see \archivedurl{https://msgpack.org/}).} Obviously, objects in the other runtime can't be persisted there. However, transferred data and objects can be recreated from persisted objects within the Smalltalk \ac{VM}.


\subsection{Moldable Interface}

While Smalltalk and the \ac{VM} are inherited from Pharo,\footnote{\emph{Cf.} \archivedurl{https://www.pharo.org/features}.} the user interface has been written afresh based on the cross-platform Skia Graphics Engine, which also powers most modern web browsers.\footnote{\emph{Cf.} \archivedurl{https://skia.org/docs/}.} Every window is rendered according to a dynamic rendering tree where every element involved (being a Smalltalk object) indicates how it wants to be laid out, and the layout is recalculated for all size changes.

\begin{cfigure}[fig_gt_screenshot]{\ac{GT} with a live notebook page (left) and inspectable object view (right)}
\includegraphics[width=.7\textwidth]{gt_screenshot}
\end{cfigure}

In its windows, \ac{GT} by default provides a tabbed interface that can show one of several tools: an object viewer, a notebook (dubbed ``Lepiter''), a code browser, a git interface and many more. While such tools are about as difficult to implement as an \ac{IDE} extension, the object viewer -- a tabbed interface itself -- is extended by simply annotating an object method that returns a \ct{GtPhlowView} object with the \ct{<gtView>} pragma as shown \eg in figure \ref{fig_annotated_view}).

\begin{cfigure}[fig_annotated_view]{Smalltalk source required for creating a custom view.}
\begin{code}
ProcessingCodeBase >> gtOutputFor: aView [
	<gtView>
	^ aView explicit
		title: 'Output' translated;
		priority: 40;
		stencil: [ (ProcessingRunner new
				limitTo: (self gtIsAnimation ifTrue: [ 30 ] ifFalse: [ 2 ]) seconds;
				run: self clone;
				canvas) asElement ]
]
\end{code}
\end{cfigure}

In this example, the element passed to the \ct{stencil:} message -- here the canvas resulting from running a Processing program -- could instead also be displayed inside a notebook page, with no annotations needed at all. Annotations are thus only required to allow \ac{GT} to discover methods of a certain type.

Similarly, methods annotated with \ct{<gtExample>} are considered tests and can be collectively inspected and run for a class or an entire package. This achieves several goals of moldable development: What starts as throw-away code can be extracted into a method and annotated, and then remains permanently available for repeated testing. Examples can also be included by name in notebooks, where they function as tested and thus guaranteed to work examples for documentation.

Since one of \ac{GT}'s stated goals is to make systems explainable \cite{Gir23}, it provides ample packages for loading, transforming, and visualizing data in various forms, such as the SmaCC parser generator,\footnote{\emph{Cf.} \archivedurl{https://refactory.com/smacc/}.} a graph builder \cite{Mey06}, \etc, but also a built-in explanation system, allowing developers to visually connect arbitrary visual elements by annotating them.\footnote{In contrast to methods, objects are annotated by sending corresponding objects: a \ct{GtExplainerTargetAptitude} or a \ct{GtExplainerExplanationAttribute}, respectively.}

What might take some getting used to: All Smalltalk code and all live objects are stored in \ac{GT}'s \ct{.image} file, which is updated whenever \ac{GT} is quit with saving.\footnote{Source code changes are additionally tracked in the \ct{.changes} journal.} This means that there are no easily accessible source files outside of \ac{GT}'s interface. Synchronization of Smalltalk code thus happens best through \ac{GT}'s built-in git client. Preexisting notebook pages are also stored within one of \ac{GT}'s subdirectories. Users can, however, create new pages in the ``Local knowledge base'',\footnote{By default, this is located in the \ct{lepiter} subdirectory of the user's documents or home folder.} which can be backed up separately and which are stored even when \ac{GT} is quit without saving. All notebook pages indicate where they are stored in their footer and can be moved between databases through that footer. This allows students to take an existing page from the teaching material and move it locally to a location that is separately backed up, in case they later delete or update \ac{GT}.

\ac{GT} was thus chosen for its moldable environment: different views are easily implemented and can be combined freely with interactions and updates between them.


\subsection{Bleeding Edge Issues}

The developers of \ac{GT} follow a trunk-only development style without release branches. This means that the release version changes almost daily, with new features being introduced gradually. This also means that subtle issues might be unexpectedly introduced in a release by or as a side-effect of some partially implemented feature. As a consequence, if \ac{GT} with an app is to be distributed, the best way to do this is by downloading the latest version, loading the app into it, verifying that it works, and then distributing \emph{this known good} image.

When \ac{GT} is used heavily, some lesser-tested code paths might be hit. We have occasionally had some modifier keys apparently lock up, requiring app switching to get keyboard shortcuts working again; we have sometimes hit a cascade of error messages, spawning dozens of debug windows, which had to be closed without other consequences; and occasionally \ac{GT} seemingly stopped responding, with even the \ct{Ctrl+.} keyboard shortcut not interrupting the running code (luckily, code modifications are backed up and restorable through the ``Code changes'' tool). Most of these are small annoyances, which more restrained users -- such as students -- shouldn't encounter often.

Finally, \ac{GT} is mainly developed under macOS and makes some platform assumptions with respect to its host operating system. This isn't noticeable when working purely within \ac{GT} but occasionally shows at its seams, with external executables not being located reliably when establishing a link to other runtimes,\footnote{\emph{Cf.} GitHub issues \href{https://github.com/feenkcom/gtoolkit/issues/4608}{feenkcom/gtoolkit\#4608} for Linux and \href{https://github.com/feenkcom/gtoolkit/issues/4633}{feenkcom/gtoolkit\#4633} for Windows.} knowledgebase names containing path separators,\footnote{Which can be worked around by renaming the database, see GitHub issue \href{https://github.com/feenkcom/gtoolkit/issues/3036}{feenkcom/gtoolkit\#3036}.} or pasting source code from third-party apps leading to visual bugs in \ac{GT}'s code editor.\footnote{This applies under Windows, see GitHub issue \href{https://github.com/feenkcom/gtoolkit/issues/4634}{feenkcom/gtoolkit\#4634}.} We assume that most of the reported issues will have been fixed at the time of reading, though.


\subsection{Historical Remarks} \label{ssc_gt_history}

Smalltalk environments have been image-based and resumable since the early days in the 1970s, when Alan Kay sketched out the original Smalltalk, which he eventually standardized at Xerox into Smalltalk-80. Based on a Smalltalk-80 \ac{VM}, Ingalls, Kay \etal started developing a new \ac{VM} and development environment at Apple that had the goal to also be customizable by non-programmers \cite{Ing97}: Squeak inherited its built-in capabilities for live and exploratory coding from the original Smalltalk, and it is back to this point that \ac{GT}'s heritage is directly tied.

While Squeak was further developed at Walt Disney Media Labs and, among other things, included in the One Laptop per Child laptops, it remained a niche product -- likely due to missing interoperability between the live environment inside its \ac{VM} and outside code. Still, Squeak and its later fork Pharo continued to be worked on and were actively used in academia and related spin-offs. Eventually, a team around Tudor G�rba set out to implement their idea of a moldable environment on the basis of Pharo, thus creating \ac{GT} \cite{Fee25}. Version 1.0 was released in 2023 and is still being actively worked on.

\ac{GT} thus has an illustrious lineage and has achieved support for many concepts asked for by literature: It is a moldable environment, supports a clean object-oriented language, allows for live and exploratory programming, still remains comparatively manageable and -- particularly relevant for this thesis -- allows for reflection at various levels, including for every object access to its method's source code, its compiled form and even its memory layout inside its \ac{VM}.

%%MAIN: thesis.tex

%%%%% The Solution %%%%%
\chapter{Proposed Solution: A New Teaching Environment} \label{ch_pa}
% formerly: \texttt{Processing Abstractions}

\section{Development of "Processing Abstractions"}
Excerpts from gt-exploration Lepiter pages

\begin{itemize}
\item Adaptable foundation: GT
\item Various approaches to run Processing: PythonBridge, interpreter, compiler, transpiler
\item Class hierarchy
\end{itemize}


\section{Abstraction Levels}
For each a short problem description and a presentation of the chosen approach:

\subsection{Source Code}
\subsection{Abstract Syntax Tree}
\subsection{Transpilation/IR}
\subsection{Machine Code}
\subsection{Output}

%%MAIN: thesis.tex

%%%%% The Validation %%%%%
\chapter{Implementation: Lesson Plans} \label{ch_teaching}
% formerly: Teaching with \texttt{PA}
In its current form, \texttt{Processing Abstractions} as presented in chapter \ref{ch_pa} is mainly targetted at the obligatory introduction to computer sciences at high school level.

Before going into empirical results from using \texttt{PA} in two courses, three lesson plans will be presented for which \texttt{PA} has been developed: a \emph{Sichtenwechsel} in computer architecture (section \ref{sc_lesson_ca}); an introduction into the inner workings of a compiler (section \ref{sc_lesson_compiler}); and a plan for a general introduction to programming (section \ref{sc_lesson_intro}). Some ideas for how to expand it for other school levels will be presented in section \ref{sc_lesson_other}.

For all the lessons, students will need a local environment of \texttt{Processing Abstractions} installed on a computer available to them. See appendix \ref{ch_setup} for how to set it up. Additionally, for non-German speaking students the contents will have to be translated to the teaching language.

\section{Lesson on Computer Architecture} \label{sc_lesson_ca}
% Using PA to demonstrate what happens under the hood when running a program in a high level language.
Introductions to computer science which extend beyond a pure programming course often contain lessons on computer architecture. E.\,g. the curriculum \cite[p.\,145]{Erz16} asks for students to ``know how computers and networks are structured and work''.

Now a sequence of lessons on the subject might be ordered either bottom up (as elaborated in subsection \ref{ssc_bottom_up}) or top down (\ref{ssc_top_down}). In either case, this proposed lesson will go towards the middle or can be used at the end as part of a repetition sequence.

\subsection{Prerequisites}
Students must already know basic programming skills in a high level language such as Processing (see section \ref{sc_processing}). In particular, they must know about variables and loops. An introduction to programming could also be done using \texttt{PA} as outlined in \ref{sc_lesson_intro} below.

The more students are supposed to work on their own, the more they'll need an overview over the different layers prior to combining them. As a prerequisite, it it recommended to at least introduce the Von Neumann architecture and its split of the CPU into control unit and arithmetic unit:

\begin{center}
\includegraphics[width=10cm]{images/Von_Neumann_Architecture.pdf}
\\ \xxx{replace or properly attribute: Kapooht, 2013, CC BY-SA 3.0}
\end{center}

In a bottom up approach, this might also include the introduction of transistors, logic gates and circuits. In a top down approach, these could also be treated afterwards.

\subsection{Lesson Plan}
The goal of the lesson is for students to have connected their knowledge of high level programming with what happens within their machine when a program is executed.

If this is the student's encounter with Glamorous Toolkit, at least a brief introduction is in order (see \ref{ssc_lesson_gt}). Else we can directly start with a reminder of what they already know about programming.


\section{Lesson on Compilers} \label{sc_lesson_compiler}
Using PA to demonstrated the steps of lexing, parsing, transpiling, compiling and optimizing.

\section{Introduction to Programming} \label{sc_lesson_intro}
Using PA as a live programming environment.

\subsection{Introduction to Glamorous Toolkit} \label{ssc_lesson_gt}

\section{Further Lesson Ideas} \label{sc_lesson_other}
Connecting PA with Smalltalk; extend it to object oriented programming; mould the environment to questions developed during the course; ...

\chapter{Validation} \label{ch_practice}
% formerly: \texttt{PA} in Practice
PA has been used twice with students (on 2025-05-12 and 2025-06-30).

\section{First Round}
\subsection{Setting}
\subsection{Observations}
\subsection{Student Feedback}
\subsection{Learnings}

\section{Second Round}
\subsection{Setting}
\subsection{Observations}
\subsection{Student Feedback}
\subsection{Learnings}


%%%%% Conclusion and Future Work %%%%%
\chapter{Conclusion}

\section{Future Work}

\appendix
\chapter{Installing and Using \texttt{Processing Abstractions}}

\chapter{Data from Questionnaires}

%END Doc
%-------------------------------------------------------

\pagebreak % References %%%%%%%%%%%%%%%%%%%%%%%%%%%%%%%%%%%%%%%%%%

% \bibliography{thesis}
\bibliographystyle{plain}

\begin{raggedright}\bibliography{thesis}\end{raggedright}
\addcontentsline{toc}{chapter}{Bibliography}

\end{document}
