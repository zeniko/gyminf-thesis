\documentclass[oneside,a4paper]{book}
%\pagestyle{headings}
\frontmatter
\input{preamble}

% A B S T R A C T
% % % % % % % % % % % % % % % % % % % % % % % % % % % % % % % % % %
\chapter*{\centering Abstract}
\begin{quotation}
\noindent

\textbf{Not an \emph{abstract} yet, but the original project description:}

Ziel des Projekts ist ein empirisch abgest�tztes Instrument f�r den Programmier-Unterricht am Gymnasium zu entwickeln, in welchem Sch�ler:innen verschiedene Abstraktionsebenen interaktiv erleben k�nnen.

Auf der Basis von Processing mit Python Syntax (https://py.processing.org/) soll der einerseits der visuelle Ablauf eines Programms, aber auch die Parsing-Schritte und die �bersetzung in Byte-Code Seite-an-Seite sicht- und untersuchbar gemacht werden, damit Sch�ler:innen die Auswirkungen ihres Programmcodes auf die Maschine live erleben k�nnen.

Die Entwicklung des Produkts wird theoretisch begleitet und das Produkt selbst empirisch gepr�ft werden.

Als Basis der Umsetzung dient Glamorous Toolkit, eine Entwicklungsumgebung basierend auf Smalltalk/Pharo, welche u.a. von Oscar Nierstrasz f�r Master- und Doktoratsstudieng�nge weiterentwickelt worden ist.

\end{quotation}
\clearpage


% C O N T E N T S
% % % % % % % % % % % % % % % % % % % % % % % % % % % % % % % % % % % % % % % %
\tableofcontents

\mainmatter
%%%%%%%%%%%%%%%%%%%%%%%%%%%%%%%%%%
%%%% NEW CHAPTER %%%%%%%%%%%%%%%%%%%%%
%%%%%%%%%%%%%%%%%%%%%%%%%%%%%%%%%%
\chapter{Introduction}
\label{cha:introduction}

\chapter {Related Work}
In which we learn what have other done to address similar problems. For example, the work of Star \cite{Star89}

\chapter{The Problem}
In which we understand what the problem is in detail.

\chapter {The Solution}
In which you describe your solution.

\chapter {The Validation}
In which you show how well the solution works.

\chapter {Conclusion and Future Work}
In which we step back, have a critical look at the entire work, then conclude, and learn what lays beyond this thesis.

\chapter {st80-test}

\lstnewenvironment{code}{%
	\lstset{%
		% frame=lines,
		frame=single,
		framerule=0pt,
		mathescape=false
	}
}{}

take plain, verbatim code,
and translate some special characters like $\wedge$ to \ct{^}. Even tabs are handled, (which is not true for verbatim).
\begin{code}
"All handled correctly: ^ $ ' % \\ << >> _ { }"
"NB: If you !{\bf really}! want an exclamation mark you must spell it BANG"
| y |
true & false not & (nil isNil) ifFalse: [self halt].
y _ self size + super size.
#($a #a 'a' 1 1.0)
	do: [:each | Transcript 
			show: (each class name); 
                     show: ' ';
                     show: (each printString).
{ 1 + 2 . 3 \\ 4 . 1 << 3. 2 >> 5 . 1 % 2 }.
^ x < y 
\end{code}

In-line code with \verb|\ct| is typed like this \verb|\ct{1 + 2 --> 3}| and looks like this: \ct{1 + 2 --> 3}, text can follow immediately.  The ``brackets'' around \verb|\ct| can be any matching pair of characters, useful if you want \ct${ and }$ in the code.

%=============================================================
\subsection*{Special chars with $\backslash$ct}
\ct@^ ~ # $ ' % \\ << >> _ {  } ! -- --> @\\
\verb|\ct@^ ~ # $ ' % \\ << >> _ {  } ! -- --> @|

%=============================================================
\subsection*{Special conventions}

\verb$\ct{Class>>>method}$ prints as \ct{Class>>>method}.\\
\verb$\ct{3 + 4 - 5 --> 2}$ prints as \ct{3 + 4 - 5 --> 2}.


%END Doc
%-------------------------------------------------------

\bibliography{thesis}
\bibliographystyle{plain}

\end{document}
