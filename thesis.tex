\documentclass[oneside,a4paper]{book}
%\pagestyle{headings}
\frontmatter
\input{preamble}

% A B S T R A C T
% % % % % % % % % % % % % % % % % % % % % % % % % % % % % % % % % %
\chapter*{\centering Abstract}
\begin{quotation}
\noindent

This thesis presents a tool for programming classes at high school level which allows students to experience and explore interactively various abstraction layers involved in running a program.

Didactic literature suggests that exploring different abstraction layers (called \emph{Sichtenwechsel}) improves students' understanding of both programming and the inner workings of a computer system. Such multi-layered exploration is also considered a foundational idea of computer science and has to be taught, among others because many lower abstraction layers tend to leak through interfaces anyways.

On the basis of the Smalltalk environment ``Glamorous Toolkit'', an interpreter for the Processing programming language in its Python form was developed and molded with many different views: Tokenization, syntax tree building, transpilation of Processing into Smalltalk, translation to an intermediary language and eventually into Smalltalk bytecode and finally the program's actual output. These views are tied to source code and updated live for seamless exploration and can be composed in interactive teaching material.

This product has been used for various lessons for which both plans and a brief evaluation is included. The evaluation shows on very limited data that the live environment does encourage experimentation and allows for students to work at their own speed and depth, allowing them to profit at their pace. Understanding of the various layers has however not improved significantly in the short period of time the tool could be tested.

\end{quotation}
\clearpage


% C O N T E N T S
% % % % % % % % % % % % % % % % % % % % % % % % % % % % % % % % % % % % % % % %
\tableofcontents

\mainmatter

%%%%% Introduction %%%%%

\chapter{Introduction}

In our modern digitized society, the importance of computer science has grown to the point where some of its subjects are taught at schools of all levels. Whereas elementary schools focus on introducing digital, connected devices and their applications, high schools also teach fundamentals, and while programming or application use courses have been implemented for decades, broader and more theoretical courses have only recently become standard. \eg in Switzerland, in 2019 computer science became an obligatory subject for all high school students, similar to more traditional sciences.

The curricula used at high schools usually contain introductions not only to algorithms and programming but among other topics also to encodings, computer architecture, networking, and social ramifications such as privacy and security (see \eg \cite{Erz16}). Students are therefore not only taught a high-level programming language such as Python but should also develop insights into what happens at various other abstraction levels when such a program is stored and run.

One traditional approach to teaching computer architecture consists in teaching a separate assembly-like language during the introduction to computer architecture. This can happen in a more gamified fashion \eg with the app Human Resource Machine \cite{Tom15}, closer to theory with the Von Neumann Simulator \cite{Gan23} or even without mnemonics using the Little Man Computer architecture \cite{Oin25}. While all of these approaches help to show how a microprocessor might work approximately, none of them offer a direct, explorable connection to a high-level language.

In our experience, high-level programming has been quite well received with students, whereas the teaching sequence on computer architecture was less so. Programming and computer architecture also hadn't fit together as nicely as we would have liked, and processor and memory have remained a mystery for too many students.

We claim that joining the computer architecture sequences to the programming one by revealing and discussing abstraction levels involved in executing a program on a (virtual) machine will improve students' understanding of computer architecture (as well as programming). For this thesis we thus set out to create an environment and course materials for testing this claim.

In this thesis, we thus present the teaching environment ``Processing Abstractions'' (introduced in chapter \ref{ch_pa}) for experiencing and discussing abstractions encountered during programming (see \eg figure \ref{fig_screenshot_vm_execution} on page \pageref{fig_screenshot_vm_execution}), allowing students to have a \emph{Sichtenwechsel} on their own programs (\ie a change of perspective with relation to layers involved, a concept introduced in \ref{sc_sichtenwechsel}) and allowing teachers to discuss abstractions in a palpable setting.

This Processing Abstractions environment connects the high-level language Processing (based on Python and introduced in section \ref{sc_processing}) with various layers down to bytecode for the Glamorous Toolkit platform (introduced in section \ref{sc_gt}), and is targeted at high school students.

We deem such a teaching environment as being called for, if didactic literature (summarized in \ref{sc_didactic}) and currently existing \acp{IDE} are considered. In particular, we propose to use this as a basis for explicitly discussing the foundational idea of multitier abstractions with students, which appear in multiple places throughout their curriculum within computer science -- most prominently in networking and information encoding -- but also other subjects such as natural sciences, psychology, \etc This discussion is important insofar as the clean separation of abstraction levels may unexpectedly fail or ``leak'' (a concept introduced in \ref{ssc_leaky_abstractions}), exposing lower levels through seemingly irrelevant side effects such as timing or physical constraints.

For teachers, suggestions for how to include the environment in the classroom are provided in chapter \ref{ch_teaching}, with a sequence on computer architecture at the center, but flanked by two sequences on programming and compilers for embedding it. All these sequences build upon the teaching environment provided and rely on students being able to get multiple, varied views and insights into the same program, in order to experience the behind-the-scenes work or rather the details usually abstracted away in an interactive way. The students' engagement is ensured due to all input being readily modifiable with changes being immediately visible, allowing for almost frictionless exploration.

Parts of the lessons suggested have already been tested with two classes at Gymnasium Neufeld in Berne for collecting valuable feedback from students. While the sample size was too small to get statistically significant results, observations and student feedback (discussed in chapter \ref{ch_practice}) have shown that the environment works and that students are motivated by its liveness to explore the concepts provided. Whether their understanding of the abstraction levels involved have improved, could, however, unfortunately not (yet) be shown.

All of this wouldn't have been possible without the very helpful support of Prof.\,em. Oscar Nierstrasz who has finally managed to introduce me to Smalltalk and Prof. Timo Kehrer who has taken this project together with his predecessor under his wing. I also want to thank my students from the classes 27Ga and 28Ga of Gymnasium Neufeld who have worked with my prototype and given helpful feedback. Finally, many thanks go to my kids for their understanding of me having to work even during their holidays and to my wife for her endless support, which made this thesis possible in the first place.

%%MAIN: thesis.tex

%%%%% The Problem %%%%%
%%%%% Related Work %%%%%
\chapter{Leaky Abstractions when Teaching Programming} \label{ch_theory}
% On the lack of connecting high-level languages with lower-level concepts

Before introducing the product of this thesis in chapter \ref{ch_pa}, we first introduce the problem it should help solve: How abstractions involved in programming are taught.

In detail, we first introduce the concept of (leaky) abstractions in multitier architectures in section \ref{sc_abstractions}; show how these are discussed in didactic literature in sections \ref{sc_didactic}; and how common IDEs already support handling these difficulties (in section \ref{sc_ides}).


\section{Multitier Architectures and (Leaky) Abstractions} \label{sc_abstractions}
In order to handle complexities arising in both theoretical and practical computer science, subjects are split into multiple layers or tiers to be described, investigated and used separatedly. \xxx{citation needed?}

Common such multitier architectures taught at high school level are the networking stack (either the seven layered OSI model or the simplified four layered DoD architecture) or the software-hardware stack ranging from apps and hardware abstracting OS down to transistors consisting of e.\,g. silicium atoms.

\xxx{diagram of such an architecture?}

Ideally, in such architectures all layers above the layer to be investigated can be ignored (beyond what the layer will be used for) and all the layers below can be abstracted away into a nicely defined interface.

As such, programming should be possible to be done independently of hardware and even the operating system, in the same way that natural languages can be taught independently of body or mind of the students.

In his article ''The Law of Leaky Abstractions'' \citep{Spo02} introduces the concept of \emph{leaky abstractions}, claiming that for all non-trivial such architectures, details of lower layers are to some degree bound to bleed through to upper layers. In other words, in practice complex interfaces tend to be incomplete or 'leaky'.

In teaching computer science, such leaky abstractions occur repeatedly, e.\,g. when an app doesn't run on a different device (with either the OS or the processor architecture leaking); or when a document seemingly can't be saved (with either the file system or differences between apps leaking).

More specifically, in programming there are several ways of abstracting away technical details:

\begin{itemize}
\item Programming instructions consist of source code which consists of encoded bits which are stored in memory or on a drive.
\item Source code consists of tokens which are usually parsed into an abstract syntax tree (AST) which are either directly or via intermediary representations translated into machine code to be run on a virtual or actual machine.
\item When programming instructions through the above abstractions are executed, variable values are encoded and stored in memory, function calls are tracked through a call stack, input state is continually mapped into memory and output is generated in several forms -- where e.g. textual output causes a font renderer to interpret glyph instructions for every character; or graphical output is anti-aliased before any pixel data is produced.
\end{itemize}
\xxx{citation needed?}

Of these different layers, students usually focus on turining instructions into source code and then checking the program's output -- or any error messages produced by the compiler or interpreter (see section \ref{sc_didactic}). Still, several of the lower layered abstractions might leak through, such as:

\begin{itemize}
\item Missing a stop condition in a recursive function leads to a cryptic ''Stack overflow'' error -- leaking information about the call stack.
\item If a program outputs emojis, they might look notably differently in source code and output -- leaking font rendering.
\item Similarly, programs containing emojis might have emojis garbled depending on the app used for inspecting the source code -- leaking text encoding.
\item If a program contains an endless loop, there might be neither error message nor output, so that it might wrongly seem that the computer isn't doing anything. This isn't an abstraction leak in the above sense but a related student misconception.
\end{itemize}


\section{Didactic Approaches} \label{sc_didactic}
See e.\,g. \cite{Sch11}, \cite{Mod16}, \cite{Har20} or \cite{Lee20} only focusing on one aspect

\subsection{Teaching Top Down}
Working downwards from gaming, as in \cite{Wei16}

\subsection{Teaching Bottom Up}
Running Tetris on NANDs as described in \cite{Cak17}, \cite{Nis21}


\section{Abstractions in IDEs} \label{sc_ides}
% Brief overview over views offered by common IDEs (such as VS Code) but mainly didactic ones such as Thonny (\cite{Ann15}), Mu (\cite{Tol23}), \emph{etc.}.

Integrated development environments used for programming offer a variety of different views on a program beyond its source code and its runtime output. The popular Visual Studio Code offers e.\,g. through extensions step-by-step debugging with variables and the call stack listed \cite{Mic25}. This is mirrored in most other full fledged IDEs such as PyCharm \citep{Jet25} or Eclipse \citep{Ecl25}.

And while such IDEs through appropriate extensions even allow inspecting Python bytecode, the respective views are usually overwhelming for programming novices and thus rather targetted at professional developers than high school students.

As a remedy, several teaching oriented IDEs have been developped, such as ''Code with Mu'' which offers a minimal command set and still allows runtime inspection \citep{Tol23}; or Thonny which had the goal to visualize runtime concepts beyond what IDEs offered at the time \citep[p. 119]{Ann15}:

On the one hand, Thonny shows intermediary steps during expression evaluation. This demonstrates that statements are not evaluated in one go, but indeed in a predetermined order operation by operation.\footnote{In professional IDEs, intermediary results are usually available by hovering over a specific operator with the order of evaluation being left to the user to determine.}

On the other hand, Thonny visualizes recursion by showing code in a new pop-up for every function call, so that multiple recursive function calls lead to an equivalent number of visible pop-ups. Most other IDEs rather show a call stack as in a separate view, which abstracts the stack into a list.\footnote{As a compromise, Glamorous Toolkit presented in chapter \ref{ch_background} displays the call stack as a list of expandable method sources with the call location highlighted.}

Finally, Thonny distinguishes between values on the stack and on the heap, showing the pointer to the heap as the value actually pushed on the stack and in a separate view the actual object on the heap at the given address.

Thus, the Thonny IDE set out to and indeed nicely visualizes several concepts on lower runtime layers.

\cite{Jal22} has assembled a list of tools targetted at visualizing some of these concepts outside of an IDE. One noteable such alternative approach is taken by Python Tutor \citep{Pyt25} which combines a visualization of stack frames variable values as pointers and deconstructed objects.

%%%%% Background Information %%%%%

\chapter{Technical Background} \label{ch_background}

Before delving into this thesis' product, an overview of the technologies involved is given in this chapter: The environment described is implemented on the Glamorous Toolkit platform, which will be introduced in \ref{sc_gt}, using Moldable Development patterns (a concept introduced in \ref{sc_moldable}). Finally, as a teaching language, we have chosen Processing for which an overview is given in \ref{sc_processing}.



\section{Processing} \label{sc_processing}

Processing is a programming language consisting of a graphics \ac{API} built upon a mainstream language as a base. Development started between 1997 and 2004 at the MIT Media Lab as a continuation of the Design By Numbers project with the goal of creating a unified environment for teaching art students the fundamentals of programming as a basis for creating digital, visual art.

While the original Design By Numbers integrated the language into an \ac{IDE}, having input and output side by side, it used its own, simplified programming language \cite{DBN01}. Processing's authors, Reas and Fry, based their language upon then popular and portable Java, removing much of the boilerplate required for object orientation, enhancing it with visual primitives and implicitly showing an output window, allowing for quick results (see figure \ref{fig_alpinerWanderweg}).

\begin{cfigure}[fig_alpinerWanderweg]{Example code (with Java syntax) and output}
\begin{minipage}{.5\textwidth}
\begin{code}
// Output canvas dimensions
size(200, 200);
// (Default white) square
rect(50, 50, 100, 100);
// Red inner rectangle
fill(255, 0, 0);
rect(50, 50 + 100 / 3, 100, 100 / 3);
\end{code}
\end{minipage}
\begin{minipage}{.45\textwidth}
\centering
\includegraphics[height=3cm]{alpinerWanderweg}
\end{minipage}
\end{cfigure}

Inside the \ac{IDE}, Processing code is compiled to Java bytecode and run inside the same \ac{JVM} as the \ac{IDE}. The Processing \ac{API} was thus provided in the form of compiled Java code, and this hasn't changed for the official Processing \ac{IDE} to this day.

Apart from graphical primitives, Processing features an implicit event loop, which allows for creating (interactive) animations within a dozen lines of code (see figure \ref{fig_jumpingBall}). Reacting to input happens either by polling while painting a frame (for this there are implicit global variables such as \ct{mousePressed}, \ct{keyCode}, \etc) or by defining event handlers alongside \ct{setup} and \ct{draw} (such as \ct{mouseClicked(event)} or \ct{keyTyped(event)}).

\begin{cfigure}[fig_jumpingBall]{Example code (with Python syntax) and four output frames}
\begin{minipage}{.5\textwidth}
\begin{code}
y = 50; dy = 0

# called once after global code
def setup():
    size(100, 200)

# called repeatedly for every frame
def draw():
    global y, dy
    background(192)
    circle(50, y, 50)
    y += dy; dy += 1
    if y > height - 25:
        dy = -0.9 * dy
\end{code}
\end{minipage}
\begin{minipage}{.45\textwidth}
\centering
\includegraphics[height=3cm]{ball1}
\includegraphics[height=3cm]{ball2}
\includegraphics[height=3cm]{ball3}
\includegraphics[height=3cm]{ball4}
\end{minipage}
\end{cfigure}

Since Python has become the prevalent teaching language \cite{Cod20}, Processing has been extended with a Python mode, which uses Python as a basis, with the Processing \ac{API} being available by default and the animation loop still being implicit \cite{Pro25}.

As the official Processing \ac{IDE} remains implemented in Java, Processing's official Python mode uses the Jython library to compile the code to Java Bytecode, so that it can be run in the same way as Processing programs written in the original Java mode \cite{Jyt25}. This also gives access to most of Python's vast standard library, with the exception of a few modules that were precompiled to native code for the various platforms for speed reasons and thus had to be rewritten for or left out of Jython.

In fact, as the \ac{JVM} is sufficiently generic to be the target for a wide variety of other languages, further modes for JavaScript\footnote{Based on the Rhino compiler from \archivedurl{https://rhino.github.io/}.} or R\footnote{Based on the Renjin interpreter from \archivedurl{https://renjin.org/}.} have been added. This allows Processing and its dedicated \ac{IDE} to be used as a starting point for programming and later seamlessly transitioning to the desired language, such as pure Python, which remains part of the motivation for students: learning an ``actually useful'' language.

Since developers have started moving away from the \ac{JVM}, there are now several reimplementations of Processing, such as p5.js for running Processing on top of JavaScript in a web environment,\footnote{\emph{Cf.} \archivedurl{https://p5js.org/} and try it out at \url{https://editor.p5js.org/}.} p5.py for running Processing in a pure Python environment,\footnote{Requiring two additional lines: \ct{from p5 import *} at the top and \ct{run()} at the bottom; \emph{cf.} \url{https://github.com/gromko/p5-python}.} or a version of Processing for microcontrollers such as Arduino.\footnote{\emph{Cf.} \archivedurl{https://www.arduino.cc/education/visualization-with-arduino-and-processing/}.} With this thesis, a limited version for a Smalltalk environment is also available (see chapter \ref{ch_pa} and appendix \ref{app_api} for an \ac{API} overview).

In fact, when we started teaching programming in high school classes, we initially ran our own \ac{IDE} based on web technologies and p5.js with custom error handling and support for live programming\footnote{This \ac{IDE} is still available at \archivedurl{https://software.zeniko.ch/ProcessingIDE.zip}. Note that it is targeted at \ct{mshta.exe} and, as such, runs best under Microsoft Windows.} before changing to the official \ac{IDE} for its Python mode. Our experience of working with Processing with ninth and tenth graders over the past decade has shown that it allows novice programmers to learn enough of the language within a month that they are able to write a clone of a game like Pong, Flappy Bird or Geometry Dash as a group project. Feedback from the various student groups about this part of the computer science curriculum has always been positive to very positive (and remains so, as we will see in chapter \ref{ch_practice}).

Reasons to use Processing are thus manifold: Processing allows a top-down approach starting with visual art, which allows teachers to motivate students with less interest in mathematics and natural sciences. Furthermore, it quickly yields pleasant-looking results, which also adds to the initial motivation \cite{Chi23}. Additionally, Processing has a large community sharing sketches and ideas, which can be used as inspiration for both students and teachers. Then, it can be based on currently widely used languages such as Python, which allows using it as a stepping stone and makes it a ``real'' programming language in the eyes of novices. In contrast, Processing itself is sufficiently unknown that even students already experienced with programming will have something new to discover. Finally, it has proven itself in our own experience over the years.



\section{Moldable Development} \label{sc_moldable}

Moldable development is a term coined by Chi\c{s}, Nierstrasz and G�rba \cite{Chi15,Chi16,Gir22} for a software development approach that should make it easier to understand a computer system by extending (``molding'') it with views and features. The goal of moldable development is to quickly get feedback on code and objects being worked on, so that a programmer can confidently make appropriate changes.

In traditional \acp{IDE}, a running system is inspected either through its source code or its live runtime objects. Available views (see \ref{ssc_ides}) are static, and new views are added through non-trivial extensions. Moldable development works in an environment in which a tool is more easily adaptable to data, making it simple to write either one-off throw-away views and tools, but also allowing developers to refactor such throw-away code into reusable components when needed.

Moldable development is thus a form of exploratory programming (cf. \ref{ssc_exploration}) on live objects where tools, whether one-off or reusable, are created in a bottom-up approach with immediate feedback available at every step. See \eg figure \ref{fig_moldable_screenshot} where the compilation of a Processing program is explored for what information about the produced bytecode to show with the goal of having one-off code sufficiently generalized that it can be added as a reusable view for all objects of this type (as eventually seen in figure \ref{fig_gt_screenshot}). Such exploration code could also later be extracted into tests, ensuring that what worked once will continue to work.

\begin{cfigure}[fig_moldable_screenshot]{The moldable \acs{GT} environment with data structures being explored for creating a view of an aspect (here: bytecode for a Smalltalk method)}
\includegraphics[width=.7\textwidth]{moldable_screenshot}
\end{cfigure}

In order to support this, a moldable environment must have extensibility in its core, allowing tools and views to be registered \eg through a simple code annotation of a few characters, which the environment can use to detect and include them (instead of having to write a lot of configuration boilerplate and overhead, which \ac{IDE} extensions meant for independent distribution usually involve).

Nierstrasz and G�rba \cite{Nie24} identified several development patterns that are common to or supportive of moldable development. One core pattern of moldable development is the ``Moldable Object'': Objects should be implementable incrementally with live object states and previously developed views remaining available throughout the whole process. An object initially consisting of little more than a data wrapper is thus extended with new functionality as it fits the available live data -- instead of designing an object on a clean slate or along tests. Extending objects iteratively based on actual needs should also ensure that code is cleanly separated, as during the exploration phase, it should become clear where code fits best.

Having a moldable environment also allows for working on code and documentation intertwinedly, similar to literate programming \cite{Knu84}. In contrast to literate programming, where code has to be extracted first, in moldable development, every code snippet should be runnable on its own, and besides code and documentation, also live results can be included. This allows a moldable environment to be used to either first document ideas and then add matching code but also to document progress or explain written code (which can then easily be extracted into a test case).

For students, such a Project Diary pattern could be used as a learning journal (similar to Microsoft OneNote\footnote{\emph{Cf.} \archivedurl{https://www.microsoft.com/de-ch/microsoft-365/onenote/digital-note-taking-app}.}), for project exploration (similar to Jupyter notebooks\footnote{\emph{Cf.} \archivedurl{https://docs.jupyter.org/}.}), or for project documentation. Another useful pattern for teaching is the ``Composed Narrative'' that visualizes object relations through side-by-side views tailored towards explaining a relation or interaction.



\section{Glamorous Toolkit} \label{sc_gt}

\acf{GT} is a fully programmable environment optimized for moldable development (see \ref{sc_moldable}), consisting of a Smalltalk \ac{VM} and runtime environment, a custom user interface, and the source code of the Smalltalk code for everything running within it. By default, it persists its entire state into a system image, so that live objects don't have to be recreated at restart and may outlive their original source \cite{Gir23}.


\subsection{Smalltalk VM}

Smalltalk is a fully object-oriented language based on message passing\footnote{One of many characteristics that Smalltalk shares with Java.} that was originally designed for educational use and, as a consequence, has rather minimalist syntax that is supposed to read more naturally: its syntax limits the need for parentheses, aligns punctuation with natural language (using full stops to end a statement and semicolons to continue a statement by sending another message to the same object) and interweaves a message's name with its arguments\footnote{\emph{Cf.} \eg the \ct{#ifTrue:ifFalse} message in figure \ref{fig_annotated_view} where each argument follows part of the name. Note that these are not the argument's names; those are declared separately in the definition of \ct{#ifTrue:ifFalse}.} \cite{Gol83}.

One potential issue for programmers experienced in ALGOL-68-derived languages is operator precedence, which in Smalltalk is limited for simplicity to just three different levels: (1) messages without arguments; (2) binary operators (which in contrast to mathematics and most other languages discussed in this thesis all have the same precedence and left associativity, and are of course also implemented as messages); and (3) all other messages.

At \ac{GT}'s core is the OpenSmalltalk Cog \ac{VM}.\footnote{\emph{Cf.} \url{https://github.com/OpenSmalltalk/opensmalltalk-vm}.} The Cog \ac{VM} is open source (MIT licensed) and shared with other Smalltalk based environments, in particular \ac{GT}'s predecessors (see \ref{ssc_gt_history}). Its source code is written in a subset of the Smalltalk language \cite{Ing97}, which is transpiled to C both for performance and for achieving cross-platform compatibility by relying on the various available C compilers. As a consequence, \ac{GT} runs on Unix systems just as well as under Microsoft Windows.

Smalltalk and the Cog \ac{VM} are highly reflective, allowing access to all but the most fundamental built-ins. In fact, all messages passed are primarily implemented in Smalltalk, but common operations can be forwarded to native code with a \ct{<primitive:...>} pragma annotation, with a fallback being provided in Smalltalk in case the native implementation fails. In particular, the execution context and the compiled bytecode of any message are available for inspection and modification. This allows users to customize the environment entirely to their liking.

For performance, the Cog \ac{VM} includes a \ac{JIT} for compiling methods called multiple times to native code on the fly \cite{Ope25}. More recently, for further optimizations, an adaptive optimizer named ``Speculative Inlining Smalltalk Architecture'' (SISTA) has been introduced by Cl�ment B�ra \cite{Ber17}, which also enables saving the optimized methods into the image, thus persisting them between restarts of the \ac{VM}. The bytecodes used for \ac{GT} are thus those proposed by B�ra and Miranda \cite{Ber14} and diverge to some extent from the original Smalltalk-80 bytecode format \cite[p.\,596]{Gol83}, in particular by enabling (more) multi-byte instructions that allow compilers to inline more common objects and code.

With \ac{GT} being based on a Smalltalk \ac{VM}, Smalltalk is \ac{GT}'s primary language. Support for other popular languages such as Python, JavaScript, or Java is, however, possible by connecting to an external runtime through the \ct{LanguageLink} protocol, \ie by passing serialized objects over sockets \cite{Fra24}.\footnote{The serialization format chosen is either JSON or the more compact binary representation MessagePack (see \archivedurl{https://msgpack.org/}).} Obviously, objects in the other runtime can't be persisted there. However, transferred data and objects can be recreated from persisted objects within the Smalltalk \ac{VM}.


\subsection{Moldable Interface}

While Smalltalk and the \ac{VM} are inherited from Pharo,\footnote{\emph{Cf.} \archivedurl{https://www.pharo.org/features}.} the user interface has been written afresh based on the cross-platform Skia Graphics Engine, which also powers most modern web browsers.\footnote{\emph{Cf.} \archivedurl{https://skia.org/docs/}.} Every window is rendered according to a dynamic rendering tree where every element involved (being a Smalltalk object) indicates how it wants to be laid out, and the layout is recalculated for all size changes.

\begin{cfigure}[fig_gt_screenshot]{\ac{GT} with a live notebook page (left) and inspectable object view (right)}
\includegraphics[width=.7\textwidth]{gt_screenshot}
\end{cfigure}

In its windows, \ac{GT} by default provides a tabbed interface that can show one of several tools: an object viewer, a notebook (dubbed ``Lepiter''), a code browser, a git interface and many more. While such tools are about as difficult to implement as an \ac{IDE} extension, the object viewer -- a tabbed interface itself -- is extended by simply annotating an object method that returns a \ct{GtPhlowView} object with the \ct{<gtView>} pragma as shown \eg in figure \ref{fig_annotated_view}).

\begin{cfigure}[fig_annotated_view]{Smalltalk source required for creating a custom view.}
\begin{code}
ProcessingCodeBase >> gtOutputFor: aView [
	<gtView>
	^ aView explicit
		title: 'Output' translated;
		priority: 40;
		stencil: [ (ProcessingRunner new
				limitTo: (self gtIsAnimation ifTrue: [ 30 ] ifFalse: [ 2 ]) seconds;
				run: self clone;
				canvas) asElement ]
]
\end{code}
\end{cfigure}

In this example, the element passed to the \ct{stencil:} message -- here the canvas resulting from running a Processing program -- could instead also be displayed inside a notebook page, with no annotations needed at all. Annotations are thus only required to allow \ac{GT} to discover methods of a certain type.

Similarly, methods annotated with \ct{<gtExample>} are considered tests and can be collectively inspected and run for a class or an entire package. This achieves several goals of moldable development: What starts as throw-away code can be extracted into a method and annotated, and then remains permanently available for repeated testing. Examples can also be included by name in notebooks, where they function as tested and thus guaranteed to work examples for documentation.

Since one of \ac{GT}'s stated goals is to make systems explainable \cite{Gir23}, it provides ample packages for loading, transforming, and visualizing data in various forms, such as the SmaCC parser generator,\footnote{\emph{Cf.} \archivedurl{https://refactory.com/smacc/}.} a graph builder \cite{Mey06}, \etc, but also a built-in explanation system, allowing developers to visually connect arbitrary visual elements by annotating them.\footnote{In contrast to methods, objects are annotated by sending corresponding objects: a \ct{GtExplainerTargetAptitude} or a \ct{GtExplainerExplanationAttribute}, respectively.}

What might take some getting used to: All Smalltalk code and all live objects are stored in \ac{GT}'s \ct{.image} file, which is updated whenever \ac{GT} is quit with saving.\footnote{Source code changes are additionally tracked in the \ct{.changes} journal.} This means that there are no easily accessible source files outside of \ac{GT}'s interface. Synchronization of Smalltalk code thus happens best through \ac{GT}'s built-in git client. Preexisting notebook pages are also stored within one of \ac{GT}'s subdirectories. Users can, however, create new pages in the ``Local knowledge base'',\footnote{By default, this is located in the \ct{lepiter} subdirectory of the user's documents or home folder.} which can be backed up separately and which are stored even when \ac{GT} is quit without saving. All notebook pages indicate where they are stored in their footer and can be moved between databases through that footer. This allows students to take an existing page from the teaching material and move it locally to a location that is separately backed up, in case they later delete or update \ac{GT}.

\ac{GT} was thus chosen for its moldable environment: different views are easily implemented and can be combined freely with interactions and updates between them.


\subsection{Bleeding Edge Issues}

The developers of \ac{GT} follow a trunk-only development style without release branches. This means that the release version changes almost daily, with new features being introduced gradually. This also means that subtle issues might be unexpectedly introduced in a release by or as a side-effect of some partially implemented feature. As a consequence, if \ac{GT} with an app is to be distributed, the best way to do this is by downloading the latest version, loading the app into it, verifying that it works, and then distributing \emph{this known good} image.

When \ac{GT} is used heavily, some lesser-tested code paths might be hit. We have occasionally had some modifier keys apparently lock up, requiring app switching to get keyboard shortcuts working again; we have sometimes hit a cascade of error messages, spawning dozens of debug windows, which had to be closed without other consequences; and occasionally \ac{GT} seemingly stopped responding, with even the \ct{Ctrl+.} keyboard shortcut not interrupting the running code (luckily, code modifications are backed up and restorable through the ``Code changes'' tool). Most of these are small annoyances, which more restrained users -- such as students -- shouldn't encounter often.

Finally, \ac{GT} is mainly developed under macOS and makes some platform assumptions with respect to its host operating system. This isn't noticeable when working purely within \ac{GT} but occasionally shows at its seams, with external executables not being located reliably when establishing a link to other runtimes,\footnote{\emph{Cf.} GitHub issues \href{https://github.com/feenkcom/gtoolkit/issues/4608}{feenkcom/gtoolkit\#4608} for Linux and \href{https://github.com/feenkcom/gtoolkit/issues/4633}{feenkcom/gtoolkit\#4633} for Windows.} knowledgebase names containing path separators,\footnote{Which can be worked around by renaming the database, see GitHub issue \href{https://github.com/feenkcom/gtoolkit/issues/3036}{feenkcom/gtoolkit\#3036}.} or pasting source code from third-party apps leading to visual bugs in \ac{GT}'s code editor.\footnote{This applies under Windows, see GitHub issue \href{https://github.com/feenkcom/gtoolkit/issues/4634}{feenkcom/gtoolkit\#4634}.} We assume that most of the reported issues will have been fixed at the time of reading, though.


\subsection{Historical Remarks} \label{ssc_gt_history}

Smalltalk environments have been image-based and resumable since the early days in the 1970s, when Alan Kay sketched out the original Smalltalk, which he eventually standardized at Xerox into Smalltalk-80. Based on a Smalltalk-80 \ac{VM}, Ingalls, Kay \etal started developing a new \ac{VM} and development environment at Apple that had the goal to also be customizable by non-programmers \cite{Ing97}: Squeak inherited its built-in capabilities for live and exploratory coding from the original Smalltalk, and it is back to this point that \ac{GT}'s heritage is directly tied.

While Squeak was further developed at Walt Disney Media Labs and, among other things, included in the One Laptop per Child laptops, it remained a niche product -- likely due to missing interoperability between the live environment inside its \ac{VM} and outside code. Still, Squeak and its later fork Pharo continued to be worked on and were actively used in academia and related spin-offs. Eventually, a team around Tudor G�rba set out to implement their idea of a moldable environment on the basis of Pharo, thus creating \ac{GT} \cite{Fee25}. Version 1.0 was released in 2023 and is still being actively worked on.

\ac{GT} thus has an illustrious lineage and has achieved support for many concepts asked for by literature: It is a moldable environment, supports a clean object-oriented language, allows for live and exploratory programming, still remains comparatively manageable and -- particularly relevant for this thesis -- allows for reflection at various levels, including for every object access to its method's source code, its compiled form and even its memory layout inside its \ac{VM}.

%%MAIN: thesis.tex

%%%%% The Solution %%%%%
\chapter{Proposed Solution: A New Teaching Environment} \label{ch_pa}
% formerly: \texttt{Processing Abstractions}

\section{Development of "Processing Abstractions"}
Excerpts from gt-exploration Lepiter pages

\begin{itemize}
\item Adaptable foundation: GT
\item Various approaches to run Processing: PythonBridge, interpreter, compiler, transpiler
\item Class hierarchy
\end{itemize}


\section{Abstraction Levels}
For each a short problem description and a presentation of the chosen approach:

\subsection{Source Code}
\subsection{Abstract Syntax Tree}
\subsection{Transpilation/IR}
\subsection{Machine Code}
\subsection{Output}

%%MAIN: thesis.tex

%%%%% The Validation %%%%%
\chapter{Implementation: Lesson Plans} \label{ch_teaching}
% formerly: Teaching with \texttt{PA}
In its current form, \texttt{Processing Abstractions} as presented in chapter \ref{ch_pa} is mainly targetted at the obligatory introduction to computer sciences at high school level.

Before going into empirical results from using \texttt{PA} in two courses, three lesson plans will be presented for which \texttt{PA} has been developed: a \emph{Sichtenwechsel} in computer architecture (section \ref{sc_lesson_ca}); an introduction into the inner workings of a compiler (section \ref{sc_lesson_compiler}); and a plan for a general introduction to programming (section \ref{sc_lesson_intro}). Some ideas for how to expand it for other school levels will be presented in section \ref{sc_lesson_other}.

For all the lessons, students will need a local environment of \texttt{Processing Abstractions} installed on a computer available to them. See appendix \ref{ch_setup} for how to set it up. Additionally, for non-German speaking students the contents will have to be translated to the teaching language.

\section{Lesson on Computer Architecture} \label{sc_lesson_ca}
% Using PA to demonstrate what happens under the hood when running a program in a high level language.
Introductions to computer science which extend beyond a pure programming course often contain lessons on computer architecture. E.\,g. the curriculum \cite[p.\,145]{Erz16} asks for students to ``know how computers and networks are structured and work''.

Now a sequence of lessons on the subject might be ordered either bottom up (as elaborated in subsection \ref{ssc_bottom_up}) or top down (\ref{ssc_top_down}). In either case, this proposed lesson will go towards the middle or can be used at the end as part of a repetition sequence.

\subsection{Prerequisites}
Students must already know basic programming skills in a high level language such as Processing (see section \ref{sc_processing}). In particular, they must know about variables and loops. An introduction to programming could also be done using \texttt{PA} as outlined in \ref{sc_lesson_intro} below.

The more students are supposed to work on their own, the more they'll need an overview over the different layers prior to combining them. As a prerequisite, it it recommended to at least introduce the Von Neumann architecture and its split of the CPU into control unit and arithmetic unit:

\begin{center}
\includegraphics[width=10cm]{images/Von_Neumann_Architecture.pdf}
\\ \xxx{replace or properly attribute: Kapooht, 2013, CC BY-SA 3.0}
\end{center}

In a bottom up approach, this might also include the introduction of transistors, logic gates and circuits. In a top down approach, these could also be treated afterwards.

\subsection{Lesson Plan}
The goal of the lesson is for students to have connected their knowledge of high level programming with what happens within their machine when a program is executed.

If this is the student's encounter with Glamorous Toolkit, at least a brief introduction is in order (see \ref{ssc_lesson_gt}). Else we can directly start with a reminder of what they already know about programming.


\section{Lesson on Compilers} \label{sc_lesson_compiler}
Using PA to demonstrated the steps of lexing, parsing, transpiling, compiling and optimizing.

\section{Introduction to Programming} \label{sc_lesson_intro}
Using PA as a live programming environment.

\subsection{Introduction to Glamorous Toolkit} \label{ssc_lesson_gt}

\section{Further Lesson Ideas} \label{sc_lesson_other}
Connecting PA with Smalltalk; extend it to object oriented programming; mould the environment to questions developed during the course; ...

\chapter{Validation} \label{ch_practice}
% formerly: \texttt{PA} in Practice
PA has been used twice with students (on 2025-05-12 and 2025-06-30).

\section{First Round}
\subsection{Setting}
\subsection{Observations}
\subsection{Student Feedback}
\subsection{Learnings}

\section{Second Round}
\subsection{Setting}
\subsection{Observations}
\subsection{Student Feedback}
\subsection{Learnings}

%%%%% The Validation %%%%%

\chapter{Validation} \label{ch_practice}

The contents produced for this thesis have been used with two courses. Evaluation of the feedback provided by students show that they did like working with the provided environment and that their understanding of the abstractions involved in programming have increased somewhat. Due to limitations in the study setting, however further analysis will be required in order to generalize these findings.

Both courses were held with classes at Gymnasium Neufeld which we have been teaching ourselves for one and two years respectively. These classes consist of Swiss high school students at ninth and tenth grade (of twelve grades) respectively.


\section{First Round} \label{sc_validation_ca} % 2025-05-12

\subsection{Setting}

The first evaluation round took place in a computer science class consisting of 17 tenth grad students. This class had at this point encountered most of the base curriculum of computer science as required in the Canton of Berne \cite[p.\,145--146]{Erz16} with only the introduction to systems architecture missing.

Specifically, the introduction to programming had happened using Processing with Python syntax inside the Processing \ac{IDE}, so students had already the desired prior experience in programming with Processing. Additionally, the class had written their last partial exam five weeks prior which had among others contained a repetition sequence on binary numbers and encodings.

Contrary to the suggestions in \ref{sc_lesson_ca}, the introduction to computer architecture was happening bottom up, loosely following Nisan and Schocken's ideas \cite{Nis21} in a significantly abbreviated course of only 6 instead of 12 weeks.\footnote{The reason for this was originally the same as Nisan and Schocken's, i.\,e. building concepts on a solid foundation.}

At the point where the environment developed for this thesis was introduced, students had already encountered some of the foundational building blocks of a modern microchip: transistors built out of semiconductors, logic gates and circuits up to an adder circuit as the basis of an \ac{ALU}.

The plan was thus to connect the already encountered foundation with their knowledge about programming by revealing and discussing several of the involved abstraction layers. Due to time constraints, only an excerpt of the sequence proposed in \ref{sc_lesson_ca} could be realized.

At the day of the lessons, 14 of the 17 students were present. At the start of the first lesson, \ac{GT} was distributed through the school's OneDrive infrastructure. The \ac{GT} environment was then introduced similar to \ref{ssc_lesson_gt} but, as was the rest of the lessons, mostly in a self-guided way with instructions being provided in \ac{GT} notebook pages.\footnote{The exact state of the materials is available at \url{https://github.com/zeniko/processing-abstractions/tree/thesis} in commit \ct{71047704f7f70c13d3d01ac520618e15d569274f} of May 12th.}

During the lessons, students were supported where needed but left to work at their own individual pace which the reduced number of students allowed for. Afterwards, a questionnaire was distributed to students for receiving their own feedback in addition to the collected observations.

At least one of the students missing the lessons was successfully able to work on the provided contents on her own.


\subsection{Observations}

During the time available, students have been able to work on their own for large amounts of time with only few common issues occurring. The interactive notebook pages seemed to allow for creating an effective teaching environment.

Additionally, many students have been observed to actively tinker with the interactive elements, as was desired and was to be expected from providing an environment for live programming and exploration. Except for a few hickups where the \ac{GT} notebook pages stopped updating (for which the usual cure of ``reloading'' helped), the interactivity worked reliably -- up to the point, where students found it so engaging that they got sidetracked by writing and modifying programs for their effect instead of the changes to the views for different layers.

Nonetheless, the more active students have been able to work through the subject matter on their own, whereas less interested students had to be motivated from time to time to continue reading and interacting. With students being able to work on their own, we had ample time for supporting these students with instructions, hints and some motivating background information.

Despite their prior Processing knowledge, students were sometimes out of their depth when changes to a Processing program were asked for. In a next round, this sequence would have to be placed closer to a programming sequence with Processing or at least a brief repetition of just using Processing would be helpful.

What caused most issues was the way \ac{GT} opened notebook pages from content links in a new page adjacent to the previous one, hiding the table of contents in the process, instead of opening them in place of the previous page as students were used to from webbrowsers. This caused students to lose track of on what pages they were supposed to be working, to the point of occasionally skipping part of the assigned content. This happened despite the brief introduction into working with \ac{GT} where closing additional pages and getting the table of contents back was an explicit introductionary task.

The overall impression of the lessons was that students had been working productively, mostly autonomously and at their own pace with the provided teaching materials.


\subsection{Student Feedback}

In order to verify our own observations, students were provided with a questionnaire for providing feedback. Of the 17 students, however only 11 returned feedback despite frequent reminding. The following yields thus at best qualitative results.\footnote{Questionnaire data in anonymized for is available at \url{https://github.com/zeniko/gyminf-thesis/blob/main/data/data_6_1.csv}.}

Additionally, three weeks later, the students have written another partial exam with individual tasks referring back to the lessons with \ac{GT}.

Student feedback shows the following: Students quite liked working with the provided environment (grading it in mostly 4/5) and reported that it worked reasonably well but not yet perfect (most students grading it either 3/5 or 4/5). This is consistent with our own observations.

Part of the reason for their liking working this way might be due to their noting programming (and one game programming project one year ago) as what they liked most about their computer science class with half the students naming this their ``highlight''.

When asked explicitly about the usefulness of the various abstraction views provided, students noted that they were very useful (mostly grading it either 4/5 or 5/5). Also, a majority of students indicated repeatedly actively interacting with the program samples.

What they liked the most was being able to work at their own speed (and optionally being able to decide for themselves whether to work together or alone) which is due to guidance from the environment which allows to introduce interactive tasks in place which students may explore autonomously in order to understand. One student explicitly noted that being able to see changes reflected instantly was gratifying.

Their main concern was two of the more engaged students noting that some explanations were not yet as clear as they could be, requiring them to ask instead of being able to work for themselves. Additionally, the quickest working student would have preferred fewer links within notebook pages, reducing the annoyance of losing the table of contents from webbrowser habits.

Students gave their feedback between one day and one week after the lesson. When asked to reword the learned content in their own words, most of them failed to describe what they'd learned entirely correctly. Would there answers have been graded, all but one students would have only been awarded at most half the points. Also, the desired connection with the previously taught contents about transistors and logic gates remained unclear with this time all students getting at most half the points.

Finally, the partial exam resulted in students answering questions related to these lessons only about 46\% correctly. The result does however weakly correlate with students' general commitment (measured by their final grade, yielding a rang correlation of $0.48$).


\subsection{Learnings}

The desired effects of having students better understand abstraction layers seems to not yet have been achieved. While the environment was engaging and led students to explore on their own, explanations will have to be expanded and made clearer for students to be able to fully understand what they're shown.

Part of the issue is however that time was too short and the implementation not fully fledged out, so these findings might also be due to both of these. So further analysis will be required and the environment will have to be tested at a larger scale with the other classes with better integration in the curriculum (as suggested by the author in \ref{sc_lesson_ca}).

At least some of the technical issues observed have since been remedied, although mostly by more explicitly telling students what to do when issues arise.

Also, since \ac{GT} runs purely on the student's own computers, their progress can't be observed other than by monitoring their screens. For this, either a separate progress tracker (such as \url{https://learningview.org/}) would have to be used or a sequence of short tests, not only checking for progress of reading but also of comprehension.

\begin{todo}
\item consider removing page linking in ``Unterrichtseinheiten''
\end{todo}



\section{Second Round} \label{sc_validation_compiler} % 2025-06-30

\subsection{Setting}

The second evaluation round took place in a computer science class consisting of 23 ninth grade students. This class had at that point passed about half the required contents of the base curriculum to computer science \cite[p.\,145--146]{Erz16}, including application usage, various encodings, algorithms and an introduction to programming using Processing eight weeks prior.\footnote{For timing reasons, unfortunately the programming project had to be postponed.}

Two lessons at the end of the school year could be set aside for an introduction to compilers as part of this thesis. These would usually have to be placed later in the curriculum, either together with the introduction to computer architecture or beyond.

The plan was to implement an excerpt of the course from \ref{sc_lesson_compiler}. As a quick overview, ``Human Resource Machine'' was used for introducing students to the limitations of machine language and motivating the need for compiling programs before executing them on actual hardware. Afterwards, \ac{GT} was introduced with additional stress on using the table of contents for navigation. Finally, students were asked to work through the provided contents at their own speed.\footnote{The exact state of the materials is available at \url{https://github.com/zeniko/processing-abstractions/tree/thesis} in commit \ct{6e22cddb176fdd46d410b9db40496bafa8a59c08} of June 30th.} Towards the end of the lessons, students were given time to fill out a questionnaire.

One unplanned limitation of these lessons were summer being early with high temperatures. As a consequence, only 15 of the 23 students were present for the introduction and the introduction had to be moved to a different, cooler location.


\subsection{Observations}

In general, the students worked reliably with the provided contents. In particular, this group seemed to quite naturally take notes within the \ac{GT} notebook pages, making the content their own.

Working speed again was heterogenous, but some smaller groups formed which supported each other. One student in particular volunteered repeatedly to help his peers.

Despite programming with Processing being rather fresh, fewer students seemed to interact with the sample programs provided, despite tasks asking them to do so explicitly. About half the students seemed content to observe views as static content.

This group had more problems getting \ac{GT} even to run. Even though these students had already successfully downloaded and used apps on their own, somehow despite a separate \ac{GT} launcher in the top level folder being provided, many failed to start \ac{GT} on their own.

Part of these issues might however relate to temperatures making it more difficult for students to focus.


\subsection{Student Feedback}

Of the 15 students present, 14 managed to hand in the questionnaire (with the last student's computer running out of battery). With this small number of answers, again no reasonable quantitative evaluation is possible.

The students answers show that many of them (12/15) have worked slower than expected, only learning about lexer and parser in the hour provided. Also, disappointingly only two of the 15 were at least somewhat confident that they would be able to explain the learned content to their peers.

This is consistent with students failing to answer basic questions about the need for compilation (half of them not answering or answering entirely wrong) but slightly better about the roles of lexer and parser (one third answering correctly and one third getting at least half of their answer correct).

It is difficult to judge how much of this is due to the environment not meeting expectations and how much is due to summer, as when asked directly about their thoughts of the provided environment, students wrongly referred ot the latter than the former.

The only general remark where students agreed on was that they had enjoyed programming (11/15) which however in contrast to the other class (see \ref{sc_validation_ca}) did not result in as much tinkering (with only 4/15 students reportedly playing around and exploring).


\subsection{Learnings}

Unfortunately, not much was to be learned from this test, mostly due to environmental factors. A repetition of this setting will thus be required at a later time.

At least the content that students actually got to seems to have been sufficiently clear for them to somewhat understand.

\begin{todo}
\item diagrams for visualization:
\begin{itemize}
\item number/percentage of students interacting
\item successfully learned
\item ???
\end{itemize}
\item verify statistical claims (don't claim too much)
\item how to best share data? \url{https://github.com/zeniko/gyminf-thesis/blob/main/data/data_6_1.csv}?
\end{todo}

%%MAIN: thesis.tex

%%%%% Conclusion and Future Work %%%%%
\chapter{Conclusion} \label{ch_conclusion}

\section{Future Work} \label{sc_future}


\appendix

%%%%% Technical Details %%%%%

\chapter{Installing and Using ``Processing Abstractions''} \label{app_setup}

In order to set up Processing Abstractions, first download \ac{GT} from \url{https://gtoolkit.com/download/} for your platform and extract the archive's entire content.

Before running it, create a new text file called \ct{startup.st} in \ac{GT}'s top-level folder besides \ct{GlamorousToolkit.image} with the following content (access it through figure \ref{fig_startup_st_qr}):

\begin{code}
Metacello new
	repository: 'github://zeniko/\ac{GT}-exploration:thesis/src';
	baseline: 'GtExploration';
	load.
Metacello new
	repository: 'github://zeniko/processing-abstractions:thesis/src';
	baseline: 'ProcessingAbstractions';
	load.

"Hide the 'Implementation and Tests' section."
GtExplorationHomeSection studentMode: true.

"Make indenting keyboard shortcuts available to non-US-English keyboard layouts
(cf. https://github.com/feenkcom/gtoolkit/issues/3002)."
LeSnippetElement keyboardShortcuts
	at: #IndentSnippet
		put: BlKeyCombinationBuilder new alt shift arrowRight build;
	at: #UnindentSnippet
		put: BlKeyCombinationBuilder new alt shift arrowLeft build.

"Make the zoom in keyboard shortcut available to de-CH keyboard layouts
(cf. https://github.com/feenkcom/gtoolkit/issues/4624)."
TLeWithFontSize compile:
	((TLeWithFontSize methodNamed: #initializeFontSizeShortcuts) sourceCode
		copyReplaceAll: 'equal' with: 'shift minus').

"Patch unneeded addressbar out of YouTube snippet
(cf. https://github.com/feenkcom/gtoolkit/issues/4560)."
LeYoutubeReferenceElement compile:
	((LeYoutubeReferenceElement methodNamed: #updatePicture) sourceCode
		copyReplaceAll: '</iframe>'' ' with: '</iframe>''; removeChildAt: 1 ').
\end{code}

Finally, run the \ct{GlamorousToolkit} executable (under Windows and Linux it is located in the \ct{bin} subfolder). The teaching materials of Processing Abstractions are now available behind the ``Unterrichtseinheiten'' home tile.

Verify that everything works as desired, and then close and save the changes to the image.

\emph{Warning:} If you have already used \ac{GT} before, the contents of your local knowledge database will also be included in \ac{GT}'s image. Therefore, rename that folder (usually \ct{lepiter/default} in your documents folder) before starting the \ac{GT} meant for distribution, and undo the renaming before starting your own \ac{GT} instance again.

Also, since the executable \ct{GlamorousToolkit.exe} is located in a subdirectory under Windows and Linux, adding a top-level link can help students. See \url{https://github.com/zeniko/gtRunner} for a ready-to-use drop-in.

\begin{cfigure}[fig_startup_st_qr]{QR-link to the code for \ct{startup.st}, for convenience}
\href{https://github.com/zeniko/gyminf-thesis/blob/main/appendix.tex}{\includegraphics[height=2.5cm]{startup.st}}
\end{cfigure}



\chapter{GT Processing API} \label{app_api}

This appendix lists the available \ac{API} calls implemented in Processing Abstraction's partial implementation of Processing.\footnote{For comparison, the full \ac{API} or Processing's Python mode is available at \archivedurl{https://py.processing.org/reference/}.} This has been autogenerated from the \ac{GT} page ``Processing API'':

\section{Rendering}

\subsection{Setup}
\begin{description}
\item[\texttt{background(r, g, b)}] \hfill \\
	Clears the canvas and changes its color (see \ct{fill(r, g, b)}). Default is light gray (192, 192, 192).
\item[\texttt{background(gray)}] \hfill \\
	Clears the canvas and changes its color (see \ct{fill(gray)}). Default is light gray (192).
\item[\texttt{size(width, height)}] \hfill \\
	Prepares an output canvas of the given dimensions. This command must always be called first (for animations: first command in \ct{def setup}).
\end{description}

\subsection{Shapes}
\begin{description}
\item[\texttt{ellipse(x, y, dx, dy)}] \hfill \\
	Draws an ellipse with the given diameters and its center at \ct{(x; y)}.
\item[\texttt{image(image, x, y)}, \texttt{image(image, x, y, width, height)}] \hfill \\
	Renders the image loaded with \ct{loadImage(...)} at the given coordinates (and scales it to fit the given size). Arguments following after the first are identical to \ct{rect}'s. If \ct{width} and \ct{height} are not given, the image's native dimensions are used.
\item[\texttt{line(x1, y1, x2, y2)}] \hfill \\
	Draws a line from \ct{(x1; y1)} to \ct{(x2; y2)}.
\item[\texttt{loadImage(pathOrUrl)}] \hfill \\
	Loads the image from the given URL or path. The returned value is to be used with \ct{image(...)}.
Paths can be absolute or relative to either \ct{FileLocator class>>gtResource} or:
\begin{code}
Element
GtInspector newOn: FileLocator documents / 'lepiter'
\end{code}
\item[\texttt{rect(x, y, width, height)}] \hfill \\
	Draws a rectangle of the given width and height, parallel to the coordinate axes with its top left corner at \ct{(x; y)}. If an optional fifth argument is given, corners are rounded by that many pixels.
\item[\texttt{text(string, x, y)}] \hfill \\
	Renders the given \ct{string} with its baseline starting at \ct{(x; y)}.
\item[\texttt{textSize(size)}] \hfill \\
	Sets the size for rendering text in pixels. Default is 12px.
\item[\texttt{triangle(x1, y1, x2, y2, x3, y3)}] \hfill \\
	Draws a triangle with its vertices at the points \ct{(x1; y1)}, \ct{(x2; y2)} and \ct{(x3; y3)}.
\end{description}

\subsection{Colors}
\begin{description}
\item[\texttt{color(r, g, b)}, \texttt{color(gray)}] \hfill \\
	Generates a color object which can be stored in a variable also be used with \ct{fill(...)}, \ct{stroke(...)} and \ct{background(...)}.
\item[\texttt{fill(r, g, b)}] \hfill \\
	Selects the color to use for filling rendered shapes. The color is given as three values in the range of 0 to 255 (red, green and blue respectively). Default is white (255, 255, 255).
\item[\texttt{fill(gray)}] \hfill \\
	Selects the gray scale value to use for filling rendered shapes. The color is given as a single value in the range of 0 to 255 (black/dark to white/light). Default is white (255).
\item[\texttt{noStroke()}] \hfill \\
	Disables borders for future shapes. Equivalent to \ct{strokeWeight(0)}.
\item[\texttt{stroke(r, g, b)}] \hfill \\
	Selects the color to use for the borders of rendered shapes (see \ct{fill(r, g, b)}). Default is black (0, 0, 0).
\item[\texttt{stroke(gray)}] \hfill \\
	Selects the gray scale value to use for the borders of rendered shapes (see \ct{fill(gray)}). Default is black (0).
\item[\texttt{strokeWeight(weight)}] \hfill \\
	Determines the size of drawn borders in pixels. Default is 0.5px.
\end{description}

\subsection{Transforms}
\begin{description}
\item[\texttt{rotate(angle)}] \hfill \\
	Rotates all future shapes by the given angle (in \ct{radians}!) clockwise around the origin.
\item[\texttt{scale(factor)}] \hfill \\
	Linearly scales all future shapes by the given factor from the origin.
\item[\texttt{translate(x, y)}] \hfill \\
	Moves the origin \ct{(0; 0)} for all future shapes (defaults to the upper left corner).
\end{description}

\section{Events}
\begin{description}
\item[\texttt{def draw():}] \hfill \\
	is called repeatedly (up to \ct{frameRate} times per second) for drawing the output.
\item[\texttt{def mouseClicked():}] \hfill \\
	is called whenever a mouse button has been clicked \emph{and} released.
\item[\texttt{def mouseMoved():}] \hfill \\
	is called whenever the mouse has been moved. Alternatively query \ct{mouseX} and \ct{mouseY} in \ct{draw()}.
\item[\texttt{def mousePressed():}] \hfill \\
	is called whenever a mouse button has been pressed. Alternatively query \ct{mousePressed} in \ct{draw()}.
\item[\texttt{def mouseReleased():}] \hfill \\
	is called whenever a mouse button has been released.
\item[\texttt{def setup():}] \hfill \\
	is called once as the program starts.
\end{description}

\section{Mathematics}
\begin{description}
\item[\texttt{cos(angle)}] \hfill \\
	Returns the cosine value for the given angle (measured in radians).
\item[\texttt{int(value)}] \hfill \\
	Rounds the value to an integer.
\item[\texttt{PI}] \hfill \\
	The value of the mathematical constant p.
\item[\texttt{max(a, b)}] \hfill \\
	Returns the larger of the two values (\ct{max(a)} returns the largest value contained in the list \ct{a}).
\item[\texttt{min(a, b)}] \hfill \\
	Returns the smaller of the two values (\ct{min(a)} returns the smallest value contained in the list \ct{a}).
\item[\texttt{radians(angle)}] \hfill \\
	Converts the given angle (measured in degrees) into radians.
\item[\texttt{random(limit)}] \hfill \\
	Returns a random floating point number between 0 and limit (inclusive).
\item[\texttt{randomSeed(seed)}] \hfill \\
	Reinitializes the random generator with the given \ct{seed} number. Using the same number will result in the exact same sequence of pseudo-randomly generated numbers.
\item[\texttt{sin(angle)}] \hfill \\
	Returns the sine value for the given angle (measured in radians).
\item[\texttt{sqrt(value)}] \hfill \\
	Returns the square root of the given value.
\item[\texttt{tan(angle)}] \hfill \\
	Returns the tangent value for the given angle (measured in radians).
\end{description}

\section{Lists}
Lists are objects which provide their own methods. Note that for the following commands, \ct{list} is a variable referencing a \ct{list}.
\begin{description}
\item[\texttt{len(list)}] \hfill \\
	Returns the number of items in this list.
\item[\texttt{list.append(value)}] \hfill \\
	Appends the value to the end of the list.
(\ct{list + [value]} instead produces a \emph{new} list. \ct{list + otherList} produces a \emph{new} list out of two lists.)
\item[\texttt{list.pop()}] \hfill \\
	Removes the last item from the list and returns the removed argument.
(\ct{list[:-1]} instead produces a \emph{new} list without the last item.)
\item[\texttt{list.reverse()}] \hfill \\
	Reverses this list's items.
(\ct{list[::-1]} instead produces a \emph{new} list with its items reversed.)
\item[\texttt{list.sort()}] \hfill \\
	Sorts this list's items.
(\ct{sorted(list)} instead produces a \emph{new} sorted list.)
\end{description}

\section{Miscellanea}
\begin{description}
\item[\texttt{delay(ms)}] \hfill \\
	Waits \ct{ms} milliseconds before continuing (mainly needed for demonstration purposes).
\item[\texttt{frameRate(fps)}] \hfill \\
	Limits the frame rate of animations to a maximum of \ct{fps} frames per second. Default is 30.
\item[\texttt{height}] \hfill \\
	Contains the canvas height as set by \ct{size()}.
\item[\texttt{millis()}] \hfill \\
	Returns the number of milliseconds that have passed since the program has started.
\item[\texttt{mouseX}, \texttt{mouseY}, \texttt{mousePressed}] \hfill \\
	Contains the \ct{x}- and \ct{y}-coordinates of the mouse cursor and whether the mouse has been pressed at the start of a \ct{draw}-phase (undefined outside of \ct{draw}).
\item[\texttt{print(value)}, \texttt{println(value)}] \hfill \\
	Prints the given value into an output console (mainly for debugging and for graphic-less program).
\item[\texttt{str(value)}] \hfill \\
	Turns the value into a string, e.\,g. for concatenating several values for use with \ct{text(...)}.
\item[\texttt{width}] \hfill \\
	Contains the canvas width as set by \ct{size()}.
\end{description}




\chapter{Views} \label{app_views}

The following pages show screenshots of all the views available through \ct{ProcessingProgram}. Multiple views are combined in single screenshots. This is not only for compactness, but also the way most views are meant to be used: for comparing various aspects of the same program. Combined views are usually linked, so that interacting with one view also affects the linked view(s).

To access the code implementing a view, \ct{Alt}+click on its tab. This will also show the view's internal name. Alternatively, all views of \ct{ProcessingProgram} are also listed in \ref{app_view_names}.

\begin{cfigure}[fig_view_abstractions]{The Abstractions view with source code, \ac{AST}, bytecode, and output showing (this is the default view for \ct{ProcessingSource} and \ct{ProcessingProgram}, as it encapsulates the essence of Processing Abstractions)}
\includegraphics[width=.7\textwidth]{view_abstractions}
\end{cfigure}

\begin{cfigure}[fig_view_chars_bytes]{The Characters and Bytes views are for discussing encodings (bytes are UTF-8 encoded)}
\includegraphics[width=.7\textwidth]{view_chars_bytes}
\end{cfigure}

\begin{cfigure}[fig_view_tokens_ast]{The Tokens and \ac{AST} views are for discussing lexer and parser}
\includegraphics[width=.7\textwidth]{view_tokens_ast}
\end{cfigure}

\begin{cfigure}[fig_view_ast_tree]{The \ac{AST} as a pannable and zoomable tree, emphasizing the \ac{AST}'s tree form}
\includegraphics[width=.7\textwidth]{view_ast_tree}
\end{cfigure}

\begin{cfigure}[fig_view_transpilation]{The Transpilation view, showing Processing and Smalltalk code side by side, is for comparing two different programming languages}
\includegraphics[width=.7\textwidth]{view_transpilation}
\end{cfigure}

\begin{cfigure}[fig_view_prefix_postfix]{Source code, \ac{AST}, and a transpilation to two pseudolanguages with pure prefix and postfix notation, respectively, for discussing programming language syntax}
\includegraphics[width=.7\textwidth]{view_prefix_postfix}
\end{cfigure}

\begin{cfigure}[fig_view_ir_bytecode]{The \ac{IR} and Bytecode views show to lower-level representations of the code}
\includegraphics[width=.7\textwidth]{view_ir_bytecode}
\end{cfigure}

\begin{cfigure}[fig_view_hexdump]{The Hexdump view serializes every method into its individual bytes}
\includegraphics[width=.7\textwidth]{view_hexdump}
\end{cfigure}

\begin{cfigure}[fig_view_runsteps]{The Runsteps view is for stepping through execution and inspect variables and stack values}
\includegraphics[width=.7\textwidth]{view_runsteps}
\end{cfigure}

\begin{cfigure}[fig_view_slices]{The Slices view shows the list of objects linking Processing and Smalltalk code (see figure \ref{fig_view_transpilation} above)}
\includegraphics[width=.7\textwidth]{view_slices}
\end{cfigure}

\begin{cfigure}[fig_view_shapes]{The Shapes view displays all output shapes individually}
\includegraphics[width=.7\textwidth]{view_shapes}
\end{cfigure}

\begin{cfigure}[fig_view_raw]{The Raw view is provided by \ac{GT} and shows all variables of an instantiated object}
\includegraphics[width=.7\textwidth]{view_raw}
\end{cfigure}

\begin{cfigure}[fig_view_meta]{The Meta view is provided by \ac{GT} and shows all methods defined for the instantiated object}
\includegraphics[width=.7\textwidth]{view_meta}
\end{cfigure}



\chapter{Technical Implementation of ``Processing Abstractions''} \label{app_implementation}


\section{Repositories}

The contents of Processing Abstractions are distributed over two GitHub repositories:

The repository \ct{zeniko/gt-exploration}\footnote{\emph{Cf.} \url{https://github.com/zeniko/gt-exploration/tree/thesis}.} contains the source code for the Processing compiler, for the runtime support, and for all the views. In addition, it also contains two dozen Processing programs as examples and test cases (in \ct{ProcessingSourceExamples}), demonstrating the entire range of implemented features. Furthermore, this repository contains process documentation in a \ac{GT} notebook named ``Implementation and Tests'' on \ac{GT}'s home screen and for that also prototype implementations of the other approaches mentioned in \ref{ssc_other_approaches}.

The repository \ct{zeniko/processing-abstractions}\footnote{\emph{Cf.} \url{https://github.com/zeniko/processing-abstractions/tree/thesis}.} mainly contains teaching material for the sequences proposed in chapter \ref{ch_teaching} in a \ac{GT} notebook named ``Unterrichtseinheiten''. In contrast to the source code repository, this content is written in German, as that was the teaching language for the validation rounds in chapter \ref{ch_practice}. Programs for examples and tasks are contained in \ct{ProcessingAbstractionsExamples} and sketches in \ct{ProcessingAbstractionSceneries}.

Both repositories have a \ct{thesis} branch that has been frozen to the state described in this thesis, whereas development will continue on the \ct{main} branches.



\section{Processing/Python Snippet} \label{app_snippet}

The Processing/Python snippet implements a model-view-viewmodel pattern. The implementation seen in figure \ref{fig_uml_snippet} mostly inherits from \ac{GT}'s Python snippet just includes minor changes to the \acs{UI} such as showing ``Processing'' in the upper right corner instead of information about a connected Python instance.

The main difference to the Python snippet is in the \ct{GtProcessingCoderModel}, which offers more different execution modes for Processing code:

\begin{itemize}
\item {\small\faPlay} calls \ct{GtProcessingCoderModel>>doIt:} to create a \ct{ProcessingSource} and show its Output view. This action's shortcut is handled by \ct{GtProcessingCoderRunShortcut}.
\item {\small\faPlay}\texttt{i} calls \ct{GtProcessingCoderModel>>doItAndGoSerialized:} to create a \ct{ProcessingProgram}, extract its runsteps, and show a \ct{Stepper} showing the \ct{gtOverviewFor:} view of \ct{ProcessingRunStep} (this is a slightly reduced variant of \ct{gtAbstractionsFor}, for manageability reasons). This action's shortcut is handled by \ct{GtProcessingCoderRunStepsShortcut}.
\item {\small\faPlay}\texttt{a} calls \ct{GtProcessingCoderModel>>doItAndGo:} to create a \ct{ProcessingProgram} and show its Abstractions view. This action's shortcut is handled by \ct{GtProcessingCoderRunDetailsShortcut}.
\item \lightning\ calls \ct{GtProcessingCoderModel>>doItAndGoAsynchronous:} to create a \ct{ProcessingProgram} and call its \ct{debug} method to start a debugging session in \ac{GT}'s debugger. This action's shortcut is handled by \ct{GtProcessingCoderDebugShortcut}.
\end{itemize}

\begin{cfigure}[fig_uml_snippet]{Diagram of classes involved in \ct{LeProcessingSnippet}}
\begin{tikzpicture}

\begin{package}{Snippet}

\begin{class}[text width=6.6cm]{LeProcessingSnippet}{0, 0}
\end{class}

\begin{class}[text width=6.6cm]{LeProcessingSnippetViewModel}{0, 2}
\end{class}

\begin{class}[text width=6.6cm]{LeProcessingSnippetElement}{0, 4}
\end{class}

\begin{class}[text width=6.6cm]{GtProcessingCoderModel}{8, 0}
\end{class}

\begin{class}[text width=6.6cm]{GtProcessingCoderViewModel}{8, 4}
\operation{doIt:}
\operation{doItAndGo:}
\operation{doItAndGoAsynchronous:}
\operation{doItAndGoSerialized:}
\operation{doItAndPublish:with:}
\end{class}

\draw[umlcd style dashed line, ->] (LeProcessingSnippet) -- node[black]{asSnippetViewModel} (LeProcessingSnippetViewModel);
\draw[umlcd style dashed line, ->] (LeProcessingSnippetViewModel) -- node[black]{snippetElementClass} (LeProcessingSnippetElement);
\draw[umlcd style dashed line, ->] (LeProcessingSnippet) -- node[black, rotate=90]{newCoder} (GtProcessingCoderModel);
\draw[umlcd style dashed line, ->] (GtProcessingCoderModel) -- node[black]{asCoderViewModel} (GtProcessingCoderViewModel);

\end{package}

\end{tikzpicture}

\end{cfigure}



\section{Views} \label{app_view_names}

The following views are available as arguments for \ct{ProcessingSource>>renderLiveView:} (see page \pageref{code_embedding_view}). Figure \ref{fig_uml_views} shows an overview of all view implementors.

\begin{cfigure}[fig_uml_views]{Diagram of all views provided and their implementors (methods set in \textit{italics} are forwarded to the actual implementor)}
\begin{tikzpicture}

\begin{package}{Processing}

\begin{class}[text width=6cm]{ProcessingSource}{8, 0}
\operation[0]{gtAbstractionsFor:}
\operation[0]{gtBytecodeFor:}
\operation[0]{gtIntermediaryRepresentationFor:}
\operation[0]{gtOutputFor:}
\operation[0]{gtSourceCodeFor:}
\operation[0]{gtTranspilationFor:}
\operation[0]{gtTreeFor:}
\end{class}

\begin{class}[text width=7cm]{ProcessingProgram}{0, 9.5}
\operation{gtAbstractionsFor:}
\operation[0]{gtBytecodeFor:}
\operation{gtBytecodePlusIRFor:}
\operation{gtBytecodePlusSourceFor:}
\operation[0]{gtHexDumpFor:}
\operation[0]{gtIntermediaryRepresentationFor:}
\operation{gtIntermediaryRepresentationPlusSourceFor:}
\operation[0]{gtOutputFor:}
\operation{gtOutputPlusSourceFor:}
\operation[0]{gtOutputShapesFor:}
\operation[0]{gtSlicesFor:}
\operation{gtSourceBytesFor:}
\operation{gtSourceBytesPlusCharsFor:}
\operation{gtSourceBytesPlusSourceFor:}
\operation{gtSourceCharsFor:}
\operation{gtSourceCharsPlusSourceFor:}
\operation{gtSourceCodeFor:}
\operation{gtStepsFor:}
\operation{gtTokensFor:}
\operation{gtTokensPlusSourceFor:}
\operation{gtTokensPlusTreeFor:}
\operation[0]{gtTranspilationFor:}
\operation{gtTranspilationPlusSourceFor:}
\operation{gtTranspilationPostfixFor:}
\operation{gtTranspilationPrefixFor:}
\operation{gtTreeFor:}
\operation{gtTreeMondrianFor:}
\operation{gtTreePlusSourceFor:}
\end{class}

\begin{interface}[text width=6cm]{ProcessingCodeBase}{8, 4.5}
\operation{gtAbstractionsFor:}
\operation{gtBytecodeFor:}
\operation{gtHexDumpFor:}
\operation{gtIntermediaryRepresentationFor:}
\operation{gtOutputFor:}
\operation{gtSlicesFor:}
\operation{gtTranspilationFor:}
\end{interface}

\begin{class}[text width=6cm]{ProcessingCanvas}{8, 7.5}
\operation{asElement}
\operation{gtOutputFor:}
\operation{gtOutputShapesFor:}
\end{class}

\begin{class}[text width=6cm]{ProcessingRunStep}{8, 13}
\operation{gtAbstractionsFor:}
\operation{gtBytecodeFor:}
\operation{gtOutputFor:}
\operation{gtOverviewFor:}
\operation[0]{gtSourceCodeFor:}
\operation{gtStackFor:}
\operation[0]{gtTranspilationPlusSourceFor:}
\operation{gtVariablesFor:}
\end{class}

\begin{class}[text width=7cm]{ProcessingTranspilationSlice}{0, 13}
\operation{gtSourceCodeFor:}
\operation{gtTranspilationFor:}
\operation{gtTranspilationPlusSourceFor:}
\end{class}

\draw[umlcd style dashed line, ->] (ProcessingSource.south west) -| node[black]{$<<$forwards to$>>$} (ProcessingProgram.south);
\draw[umlcd style dashed line, ->] (ProcessingProgram.north east) -- +(0.5, 0) -- +(0.5, -9.1) -| node[black]{$<<$forwards to$>>$} (ProcessingCodeBase.south);
\draw[umlcd style dashed line, ->] (ProcessingProgram.north east) -- +(0.5, 0) -- +(0.5, -1.5) -| node[black]{$<<$forwards to$>>$} (ProcessingCanvas);
\draw[umlcd style dashed line, ->] (ProcessingCodeBase) -- node[black]{$<<$embeds$>>$} (ProcessingCanvas);
\draw[umlcd style dashed line, ->] (ProcessingProgram.north east) -- +(0.5, 0) -- +(0.5, -1) -| node[black]{$<<$embeds$>>$} (ProcessingRunStep);
\draw[umlcd style dashed line, ->] (ProcessingRunStep.west) -- +(-0.5, 0) -- +(-0.5, -0.5) -| node[black]{$<<$forwards to$>>$} (ProcessingTranspilationSlice);

\end{package}

\end{tikzpicture}

\end{cfigure}

\begin{description}
\item[\texttt{gtAbstractionsFor:}] \hfill \\
	(implemented by \ct{ProcessingProgram}) combines \ct{gtSourceCodeFor:}, \ct{gtTreeFor:}, \ct{gtBytecodeFor:}, and \ct{gtOutputFor:}, linking the source to the other three views through a common \ct{Announcer} reacting to selection changes (shown in figure \ref{fig_view_abstractions}).
\item[\texttt{gtBytecodeFor:}] \hfill \\
	uses the \ct{CompiledMethod} instances of each method in a compiled class' \ct{class methodDict} to access its \ct{symbolicBytecodes} and show the resulting \ct{SymbolicBytecode}s' bytes and mnemonic (shown on the right in figure \ref{fig_view_ir_bytecode}). \ct{gtBytecodePlusSourceFor:} combines this view with \ct{gtSourceCodeFor:}.
\item[\texttt{gtBytecodePlusIRFor:}] \hfill \\
	combines the \ct{gtBytecodeFor:} with \ct{gtIntermediaryRepresentationFor:} (shown in figure \ref{fig_view_ir_bytecode}).
\item[\texttt{gtHexDumpFor:}] \hfill \\
	shows individual bytes for all compiled methods.
\item[\texttt{gtIntermediaryRepresentationFor:}] \hfill \\
	uses the \ct{OpalCompiler} to translate the transpiled Smalltalk code to \ct{IRInstruction}s (shown on the left in figure \ref{fig_view_ir_bytecode}). \ct{gtIntermediaryRepresentationPlusSourceFor:} combines this view with \ct{gtSourceCodeFor:}.
\item[\texttt{gtOutputFor:}] \hfill \\
	displays a newly created \ct{ProcessingCanvasElement} with attached event listeners for interactivity (shown on the bottom right in figure \ref{fig_view_abstractions}). \ct{gtOutputPlusSourceFor:} combines this view with \ct{gtSourceCodeFor:}.
\item[\texttt{gtOutputShapesFor:}] \hfill \\
	shows a list of all \ct{ProcessingCanvasShape}s in the order they were drawn (shown in figure \ref{fig_view_shapes}). Clearing the canvas with \ct{background(...)} also clears this list.
\item[\texttt{gtSlicesFor:}] \hfill \\
	shows a list of \ct{ProcessingTranspilationSlice}s with the corresponding expressions in Processing source code and Smalltalk transpilation highlighted (shown in figure \ref{fig_view_slices}).
\item[\texttt{gtSourceBytesFor:}] \hfill \\
	shows a list of \ct{SmallInteger}s corresponding to the bytes of the source code after UTF-8 encoding (shown on the right in figure \ref{fig_view_chars_bytes}) \ct{gtSourceBytesPlusSourceFor:} combines this view with \ct{gtSourceCodeFor:}.
\item[\texttt{gtSourceBytesPlusCharsFor:}] \hfill \\
	combines \ct{gtSourceCharsFor:} and \ct{gtSourceBytesFor:} (shown in figure \ref{fig_view_chars_bytes}).
\item[\texttt{gtSourceCharsFor:}] \hfill \\
	shows a list of \ct{Character}s corresponding to each of the Processing source code's characters (shown on the left in figure \ref{fig_view_chars_bytes}). \ct{gtSourceCharsPlusSourceFor:} combines this view with \ct{gtSourceCodeFor:}.
\item[\texttt{gtSourceCodeFor:}] \hfill \\
	displays a fresh read-only editor instance from \ct{SmaCCParseNode>>gtSourceEditorWithHightlight:} with a custom \ct{BrTextEditorReadonlyWithNavigationMode} mode (shown on the top left in figure \ref{fig_view_abstractions}, on the left in \ref{fig_view_transpilation}, \etc).
\item[\texttt{gtStepsFor:}] \hfill \\
	shows a list of \ct{ProcessingRunStep}s and their own \ct{gtAbstractionFor:} view, consisting of their \ct{gtSourceCodeFor:}, \ct{gtBytecodeFor:}, \ct{gtVariablesFor:}, \ct{gtStackFor:}, and \ct{gtOutputFor:} (shown in figure \ref{fig_view_runsteps}). \ct{ProcessingRunStep>>gtOverviewFor:} is a simplified variant of this shown from the Processing/Python snippet.
\item[\texttt{gtTokensFor:}] \hfill \\
	shows a list of \ct{SmaCCToken}s that were produced by \ct{ProcessingParser} (shown on the left in figure \ref{fig_view_tokens_ast}). \ct{gtTokensPlusSourceFor:} combines this view with \ct{gtSourceCodeFor:}.
\item[\texttt{gtTokensPlusTreeFor:}] \hfill \\
	combines \ct{gtTokensFor:} with \ct{gtTreeFor:} (shown in figure \ref{fig_view_tokens_ast}).
\item[\texttt{gtTranspilationFor:}] \hfill \\
	shows the transpiled Smalltalk code in \ct{GT}'s code viewer, which separates methods and adds syntax highlighting. \ct{gtTranspilationPlusSourceFor:} combines this view with \ct{gtSourceCodeFor:} (shown in figure \ref{fig_view_transpilation}).
\item[\texttt{gtTranspilationPostfixFor:} and \texttt{gtTranspilationPrefixFor:}] \hfill \\
	show a \ct{String} produced by \ct{ProcessingTranspilerVariant} in either its \ct{prefix} or \ct{postfix} modes (both shown at the bottom in figure \ref{fig_view_prefix_postfix}).
\item[\texttt{gtTreeFor:}] \hfill \\
	shows a treelist of \ct{PyRootNode}s for expression roots and \ct{SmaCCToken}s for structural tokens (shown on the right in figure \ref{fig_view_tokens_ast}).
\item[\texttt{gtTreeMondrianFor:}] \hfill \\
	shows a horizontal \ct{GtMondrian} tree produced by \ct{ProcessingTreeMondrianCreator} (shown in figure \ref{fig_view_ast_tree}). \ct{gtTreePlusSourceFor:} combines this view with \ct{gtSourceCodeFor:}.
\end{description}



\chapter{Questionnaires} \label{app_questionnaires}

The following questionnaires are a reproduction of the originals, which were implemented in \href{https://forms.office.com/}{Microsoft Forms}, and are in German, since that's the students' teaching language.



\section{Questionnaire for \ref{sc_validation_ca}} \label{app_questionnaire_1}


\subsection*{Feedback zur heutigen Unterrichtssequenz}

\begin{Questionnaire}

\item \Question{Wie hat Ihnen die heutige Unterrichtssequenz gefallen?} \\ gar nicht \Qrating{5} sehr gut

\item \Question{Welche Themen haben Sie heute alle bearbeiten k�nnen?}
\begin{Qlist}
\item Arbeiten mit Glamorous Toolkit
\item Maschinensprache und Prozessor
\item Funktionen eines Compilers
\item Anhang
\end{Qlist}

\item \Question{Wie sehr trauen Sie sich zu, die heutigen Inhalte jemand anderem zu erkl�ren?} \\ gar nicht \Qrating{5} \emph{easy-peasy}

\item \Question{Wie viele der Python-Progr�mmchen haben Sie selbst ver�ndert?}
\begin{Qlist}[$\ocircle$]
\item Keines
\item Eines
\item Zwei bis drei
\item Vier oder mehr
\end{Qlist}

\item \Question{Was ist ein Stack?} \Qlines{1}

\item \Question{Was machen Lexer und Parser?} \Qlines{1}

\item \Question{Wie gut hat die Lernumgebung f�r Sie funktioniert?} \\ gar nicht \Qrating{5} problemlos

\item \Question{Was hat Ihnen an der Lernumgebung gefallen?} \Qlines{2}

\item \Question{Welche �nderungen an der Lernumgebung w�nschen Sie sich f�r die n�chste Klasse?} \Qlines{2}

\item \Question{Wie hilfreich fanden Sie die Nebeneinanderstellungen der unterschiedlichen Schritte beim Ausf�hren/�bersetzen eines Programms?} \\ weglassen \Qrating{5} bitte mehr davon

\item \Question{Beschreiben Sie in eigenen Worten: Wie wird ein Programm auf einem Prozessor ausgef�hrt?} \Qlines{2}

\item \Question{Beschreiben Sie in eigenen Worten: Wie wird ein Programm in einer Hochsprache wie Processing f�r den Prozessor aufbereitet?} \Qlines{2}

\item \Question{Was hatte das heutige Thema mit Silizium, Transistoren, Gattern und Schaltungen zu tun?} \Qlines{2}

\suspend{Questionnaire}


\subsection*{Feedback zum Informatikunterricht der letzten zwei Jahre}

\resume{Questionnaire}

\item \Question{Was war f�r Sie das Highlight vom Informatik-Unterricht? (Was hat Ihnen am meisten Eindruck gemacht?)} \Qlines{1}

\item \Question{Was ist Ihnen vom Informatik-Unterricht alles geblieben (einige Stichworte zum Stoff)?} \Qlines{2}

\item \Question{Was hat Ihnen am Informatik-Unterricht gefallen?} \Qlines{2}

\item \Question{Wenn Sie mir vor zwei Jahren einen Hinweis geben k�nnten: Was h�tten Sie sich f�r den Unterricht anders gew�nscht?} \Qlines{2}

\end{Questionnaire}



\section{Questionnaire for \ref{sc_validation_compiler}} \label{app_questionnaire_2}


\subsection*{Feedback zur heutigen Unterrichtssequenz}

\begin{Questionnaire}

\item \Question{Wie hat Ihnen die heutige Unterrichtssequenz gefallen?} \\ gar nicht \Qrating{5} sehr gut

\item \Question{Welche Themen haben Sie heute alle bearbeiten k�nnen?}
\begin{Qlist}
\item Arbeiten mit Glamorous Toolkit
\item Human Resource Machine (aus Maschinensprache und Prozessor)
\item Lexer und Parser
\item Transpiler und Compiler
\item Optimierer
\end{Qlist}

\item \Question{Wie sehr trauen Sie sich zu, die heutigen Inhalte jemand anderem zu erkl�ren?} \\ gar nicht \Qrating{5} \emph{easy-peasy}

\item \Question{Wie viele der Python-Progr�mmchen haben Sie selbst ver�ndert?}
\begin{Qlist}[$\ocircle$]
\item Keines
\item Eines
\item Zwei bis drei
\item Vier oder mehr
\end{Qlist}

\item \Question{Weshalb kann ein Prozessor ein Processing-Programm nicht ohne �bersetzung ausf�hren} \Qlines{1}

\item \Question{Was machen Lexer und Parser?} \Qlines{1}

\item \Question{Wie gut hat die Lernumgebung f�r Sie funktioniert?} \\ gar nicht \Qrating{5} problemlos

\item \Question{Was hat Ihnen an der Lernumgebung gefallen?} \Qlines{2}

\item \Question{Welche �nderungen an der Lernumgebung w�nschen Sie sich f�r die n�chste Klasse?} \Qlines{2}

\item \Question{Wie hilfreich fanden Sie die Nebeneinanderstellungen der unterschiedlichen Schritte beim Ausf�hren/�bersetzen eines Programms?} \\ weglassen \Qrating{5} bitte mehr davon

\item \Question{Beschreiben Sie in eigenen Worten: Wie wird ein Programm in einer Hochsprache wie Processing f�r den Prozessor aufbereitet?} \Qlines{2}

\item \Question{Was hatte das heutige Thema mit Codierung und was mit Programmieren zu tun?} \Qlines{2}

\suspend{Questionnaire}


\subsection*{Feedback zum Informatikunterricht}

\resume{Questionnaire}

\item \Question{Was war f�r Sie das Highlight vom Informatik-Unterricht? (Was hat Ihnen am meisten Eindruck gemacht?)} \Qlines{1}

\item \Question{Was ist Ihnen vom Informatik-Unterricht alles geblieben (einige Stichworte zum Stoff)?} \Qlines{2}

\item \Question{Was hat Ihnen am Informatik-Unterricht gefallen?} \Qlines{2}

\item \Question{Welche �nderungen w�nschen Sie sich f�rs kommende Schuljahr in Informatik?} \Qlines{2}

\end{Questionnaire}


%END Doc
%-------------------------------------------------------

\pagebreak % Abbreviations %%%%%%%%%%%%%%%%%%%%%%%%%%%%%%%%%%%%%%%%%%

\addcontentsline{toc}{chapter}{Abbreviations}
\chapter*{Abbreviations}

\begin{acronym}[REPL]
\acro{API}{Application Programming Interface}
\acro{ALU}{Arithmetic Logic Unit}
\acro{AST}{Abstract Syntax Tree}
\acro{CPU}{Central Processing Unit}
\acro{GT}{Glamorous Toolkit}
\acro{IDE}{Integrated Development Environment}
\acro{JVM}{Java Virtual Machine}
\acro{LLM}{Large Language Model}
\acro{REPL}{Read-Evaluate-Print-Loop}
\acro{VM}{Virtual Machine}
\end{acronym}

\pagebreak % References %%%%%%%%%%%%%%%%%%%%%%%%%%%%%%%%%%%%%%%%%%

\addcontentsline{toc}{chapter}{Bibliography}

\bibliographystyle{plain}
\begin{raggedright}\bibliography{thesis}\end{raggedright}

\end{document}
