\documentclass[oneside,a4paper]{book}
%\pagestyle{headings}
\frontmatter
\input{preamble}

% A B S T R A C T
% % % % % % % % % % % % % % % % % % % % % % % % % % % % % % % % % %
\chapter*{\centering Abstract}
\begin{quotation}
\noindent

\textbf{Not an \emph{abstract} yet, but the original project description:}

Ziel des Projekts ist, ein empirisch abgest�tztes Instrument f�r den Programmier-Unterricht am Gymnasium zu entwickeln, in welchem Sch�ler:innen verschiedene Abstraktionsebenen interaktiv erleben k�nnen.

Auf der Basis von Processing mit Python Syntax (https://py.processing.org/) soll der einerseits der visuelle Ablauf eines Programms, aber auch die Parsing-Schritte und die �bersetzung in Byte-Code Seite-an-Seite sicht- und untersuchbar gemacht werden, damit Sch�ler:innen die Auswirkungen ihres Programmcodes auf die Maschine live erleben k�nnen.

Die Entwicklung des Produkts wird theoretisch begleitet und das Produkt selbst empirisch gepr�ft werden.

Als Basis der Umsetzung dient Glamorous Toolkit, eine Entwicklungsumgebung basierend auf Smalltalk/Pharo, welche u.a. von Oscar Nierstrasz f�r Master- und Doktoratsstudieng�nge weiterentwickelt worden ist.

\end{quotation}
\clearpage


% C O N T E N T S
% % % % % % % % % % % % % % % % % % % % % % % % % % % % % % % % % % % % % % % %
\tableofcontents

\mainmatter
%%%%% Introduction %%%%%
\chapter{Introduction}


%%%%% The Problem %%%%%
\chapter{Teaching Programming Abstractions}
On the lack of connecting high-level languages with lower-level concepts (and \emph{leaky abstractions})

\section{Personal Experience}
Programming with Processing (by \cite{Cas14}) vs. "Little Man Computer" or "Human Resource Machine"

\section{Didactic Approaches}
See e.g. \cite{Sch11}, \cite{Mod16}, \cite{Har20} or \cite{Lee20}

\section{Limitations of IDEs}
Such as VS Code or Thonny


%%%%% Related Work %%%%%
\chapter{Didactic Approaches}

\section{Top Down}
Working downwards from gaming, as in \cite{Wei16}

\section{Bottom Up}
Running Tetris on NANDs as described in \cite{Cak17}, \cite{Nis21}


%%%%% The Solution %%%%%
\chapter {\texttt{Processing Abstractions}}

\section{Moldable Development}
Referring to \cite{Nie24}.

\section{Development of "Processing Abstractions"}
Excerpts from gt-exploration Lepiter pages

\subsection{Parsing}
\subsection{Transpiling}
\subsection{Compiling}
\subsection{Running}

%%%%% The Validation %%%%%
\chapter{\texttt{PA} in Practice}
How students reacted to using it

\section{First Round}
\subsection{Observations}
\subsection{Student Feedback}
\subsection{Learnings}

\section{Second Round}
\subsection{Observations}
\subsection{Student Feedback}
\subsection{Learnings}

%%%%% Conclusion and Future Work %%%%%
\chapter{Conclusion}

\appendix
\chapter{Installing and Using \texttt{Processing Abstractions}}

\chapter{Data from Questionnaires}

%END Doc
%-------------------------------------------------------

\pagebreak % References %%%%%%%%%%%%%%%%%%%%%%%%%%%%%%%%%%%%%%%%%%

% \bibliography{thesis}
\bibliographystyle{plain}

\begin{raggedright}\bibliography{thesis}\end{raggedright}
\addcontentsline{toc}{chapter}{Bibliography}

\end{document}
