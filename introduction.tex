%%MAIN: thesis.tex

%%%%% Introduction %%%%%
\chapter{Introduction}

In modern digitized society, the importance of computer science has grown to the point where some of its subjects are taught at schools of all levels. Whereas elementary schools focus on introducing digital, connected devices and their applications, high schools also teach fundamentals. And while programming or application use courses have been implemented since decades ago, broader and more theoretical courses have recently become standard. E.\,g. in Switzerland, computer science has become an obligatory subject for all high school students similar to more traditional sciences starting in 2019.

The curricula used at high schools usually contain introductions not only to algorithms and programming, but among others also into encodings, computer architecture, networking and social ramifications such as privacy and security (see e.\,g. \cite{Erz16}). As such, students not only are taught a high level programming language such as Python, but should also have insights into what happens at various other abstraction layers when such a program is stored and run.

One traditional approach to teaching computer architecture consists in teaching a separate assembly-like language during the introduction to computer architecture. This can happen in a more gamified fashion e.\,g. with ``Human Resource Machine'' \cite{Tom15}, closer to theory like ``Von Neumann Simulator'' \cite{Gan23} or even without mnemonics as using the Little Man Computer architecture \cite{Oin25}. While all of these approaches help to show how a microprocessor might approximately work, none of them offer a direct, explorable connection to a high level language.

One suggestion for such a direct connection between high level language and machine code will be presented in this thesis in form of a teaching environment targetted at high school students and implemented based on ``Glamorous Toolkit'' (introduced in section \ref{sc_gt}) for the ``Processing'' programming language based on Python syntax (which will be introduced in section \ref{sc_processing}).

We consider such a teaching environment being called for, if didactic literature (summarized in \ref{sc_didactic}) and currently existing IDEs are considered. In particular, we propose to use this as a basis for explicitly discussing the foundational idea of multitier abstractions with students which appear in multiple places all throughout their curriculum within computer science -- most prominently in networking and in information encoding -- but also in other subject matters such as natural sciences, psychology, \emph{etc.}. This discussion is important insofar as the clean separation of abstraction layers may unexpectedly fail or ``leak'' (a concept introduced in \ref{ssc_leaky_abstractions}).

The teaching environment introduced in chapter \ref{ch_pa} of this thesis is called \texttt{Processing Abstractions} and will focus in particular on abstractions encountered during programming (see e.\,g. figure \ref{fig_screenshot_vm_execution}), allowing students to have a \emph{Sichtenwechsel} on their own programs.

For teachers, suggestions for how to include it in the classroom are provided in chapter \ref{ch_teaching}, with a sequence on computer architecture at the center, but for embedding surrounded by two sequences on programming and compilers. All of these sequences build upon the teaching environment provided and rely on students being able to get multiple, varied views and insights into the same program, in order to experience the behind-the-scenes work or rather the details usually abstracted away in an interactive way. The student's engagement builds upon all input being readily modifiable with changes being immediately visible, allowing for almost frictionless exploration.

Parts of suggested lessons have already been implemented with two classes at the Gymnasium Neufeld in Berne for collecting valuable feedback from students. While the sample size was too small to get significant results, observations and student feedback (discussed in chapter \ref{ch_practice}) have shown that the environment works and that students are motivated by its liveness to explore the concepts provided. Whether their understanding of the abstraction levels involved have improved, could however unfortunately not (yet) be shown.

All of this wouldn't have been possible without the very helpful support of Prof.\,em. Oscar Nierstraz who has finally managed to introduce me to Smalltalk and Prof. Timo Kehrer who has taken this project together with his predecessor below his wing. I also want to thank my students from the classes 27Ga and 28Ga of Gymnasium Neufeld who have worked with my productions and given helpful feedback. Finally, many thanks go to my kids for their understanding of me having to work even during their holidays and to my wife for her endless support which made this thesis possible in the first place.
