%%MAIN: thesis.tex

%%%%% The Solution %%%%%
\chapter{Proposed Solution: A New Teaching Environment} \label{ch_pa}
% formerly: \texttt{Processing Abstractions}

\section{Development of "Processing Abstractions"}
Excerpts from gt-exploration Lepiter pages

\begin{todo}
\item Adaptable foundation: GT
\item Various approaches to run Processing: PythonBridge, interpreter, compiler, transpiler
\item Class hierarchy
\end{todo}


\section{Abstraction Levels}
For each a short problem description and a presentation of the chosen approach:

\subsection{Source Code}
\subsection{Abstract Syntax Tree}
\subsection{Transpilation/IR}
\subsection{Machine Code}
\subsection{Output}


\begin{code}
ProcessingCanvas >> ellipse: x y: y dx: dx dy: dy [
	self
		ellipse: dx
		by: dy
		at: x @ y
]
\end{code}


\begin{code}
ProcessingCanvas >> endFrame [
	presenter updateOutput.
	(1 / frameRate) seconds wait.	"The frame rate is adjustable through `frameRate()`"
	frameCount := frameCount + 1.
	transform := #yourself	"Transforms are reset at the end of a draw-cycle"
]
\end{code}
