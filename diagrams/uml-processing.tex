%%%%% The Solution (UML Diagram) %%%%%

\newgeometry{bottom=1cm}

\begin{landscape}

\begin{cfigure}[fig_uml_processing]{Diagram of most classes involved in running Processing code within \ac{GT}}

\begin{tikzpicture}

\begin{package}{Processing}

\begin{class}[text width=6cm]{ProcessingSource}{0, 0}
\operation{fromFile:}
\operation{fromPage:at:}
\operation{fromSnippet:}
\end{class}

\begin{class}[text width=6cm]{ProcessingProgram}{0, 2.5}
\operation{ast}
\operation{compilation}
\operation{run}
\end{class}

\begin{class}[text width=6cm]{ProcessingParser}{-8, 0}
\operation{parse:}
\end{class}

\begin{class}[text width=6cm]{ProcessingTranspiler}{-8, 2.5}
\operation{transpile:}
\end{class}

\begin{class}[text width=6cm]{ProcessingTranspilationSlice}{-8, 7.5}
\operation{link:method:from:to:}
\end{class}

\begin{interface}[text width=6cm]{ProcessingCodeBase}{0, 5}
\operation{gtRun}
\end{interface}

\begin{class}[text width=6cm]{ProcessingRunner}{8, 5}
\operation{run:}
\operation{runSteps:}
\end{class}

\begin{class}[text width=6cm]{ProcessingCanvas}{0, 7.5}
\operation{asElement}
\operation{presenter}
\end{class}

\begin{class}[text width=6cm]{ProcessingCanvasPresenter}{8, 7.5}
\implement{ProcessingCanvas}
\end{class}

\begin{class}[text width=6cm]{ProcessingCanvasElement}{8, 9}
\end{class}

\begin{class}[text width=6cm]{ProcessingAstCleaner}{-8, 5}
\operation{clean:}
\end{class}

\begin{interface}[text width=6cm]{ProcessingCanvasShape}{0, 9}
\end{interface}

\begin{class}[text width=6cm]{ProcessingRunStep}{8, 2.5}
\end{class}

\draw[umlcd style dashed line, ->] (ProcessingSource) -- node[black]{$<<$accesses$>>$} (ProcessingProgram);
\draw[umlcd style dashed line, <->] (ProcessingProgram) -- node[black, sloped]{$<<$source $\rightarrow$ \ac{AST}$>>$} (ProcessingParser);
\draw[umlcd style dashed line, ->] (ProcessingProgram) -- node[black]{$<<$uses$>>$} (ProcessingTranspiler);
\draw[umlcd style dashed line, ->] (ProcessingTranspiler) -- node[black, sloped]{$<<$creates$>>$} (ProcessingCodeBase);
\draw[umlcd style dashed line, <->] (ProcessingTranspiler) -- node[black]{$<<$uses$>>$} (ProcessingAstCleaner);
\draw[umlcd style dashed line, ->] (ProcessingTranspiler.west) -- +(-1, 0) |- node[black, pos=0.25, sloped]{$<<$creates$>>$} (ProcessingTranspilationSlice);
\draw[umlcd style dashed line, ->] (ProcessingProgram) -- node[black]{$<<$accesses$>>$} (ProcessingCodeBase);
\draw[umlcd style dashed line, ->] (ProcessingProgram) -- node[black, sloped]{$<<$uses$>>$} (ProcessingRunner);
\draw[umlcd style dashed line, ->] (ProcessingRunner) -- node[black]{$<<$runs$>>$} (ProcessingCodeBase);
\draw[umlcd style dashed line, ->] (ProcessingRunner) -- node[black]{$<<$creates$>>$} (ProcessingRunStep);
\draw[umlcd style dashed line, ->] (ProcessingCodeBase) -- node[black]{$<<$owns$>>$} (ProcessingCanvas);
\draw[umlcd style dashed line, ->] (ProcessingRunner) -- node[black, sloped]{$<<$creates$>>$} (ProcessingCanvas);
\draw[umlcd style dashed line, ->] (ProcessingCanvas) -- node[black]{$<<$controls$>>$} (ProcessingCanvasPresenter);
\draw[umlcd style dashed line, ->] (ProcessingCanvasPresenter) -- node[black]{$<<$interacts$>>$} (ProcessingCanvasElement);
\draw[umlcd style dashed line, ->] (ProcessingCanvas) -- node[black]{$<<$creates$>>$} (ProcessingCanvasShape);
\draw[umlcd style dashed line, ->] (ProcessingCanvasShape) -- node[black]{$<<$renders$>>$} (ProcessingCanvasElement);

\end{package}

\end{tikzpicture}

\end{cfigure}

\end{landscape}

\restoregeometry
