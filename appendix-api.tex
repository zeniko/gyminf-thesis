\section{Rendering}

\subsection{Setup}
\begin{description}
\item[\texttt{background(r, g, b)}] \hfill \\
	Clears the canvas and changes its color (see \ct{fill(r, g, b)}). Default is light gray (192, 192, 192).
\item[\texttt{background(gray)}] \hfill \\
	Clears the canvas and changes its color (see \ct{fill(gray)}). Default is light gray (192).
\item[\texttt{size(width, height)}] \hfill \\
	Prepares an output canvas of the given dimensions. This command must always be called first (for animations: first command in \ct{def setup}).
\end{description}

\subsection{Shapes}
\begin{description}
\item[\texttt{ellipse(x, y, dx, dy)}] \hfill \\
	Draws an ellipse with the given diameters and its center at \ct{(x; y)}.
\item[\texttt{image(image, x, y)}, \texttt{image(image, x, y, width, height)}] \hfill \\
	Renders the image loaded with \ct{loadImage(...)} at the given coordinates (and scales it to fit the given size). Arguments following after the first are identical to \ct{rect}'s. If \ct{width} and \ct{height} are not given, the image's native dimensions are used.
\item[\texttt{line(x1, y1, x2, y2)}] \hfill \\
	Draws a line from \ct{(x1; y1)} to \ct{(x2; y2)}.
\item[\texttt{loadImage(pathOrUrl)}] \hfill \\
	Loads the image from the given URL or path. The returned value is to be used with \ct{image(...)}.
Paths can be absolute or relative to either \ct{FileLocator class>>gtResource} or:
\begin{code}
Element
GtInspector newOn: FileLocator documents / 'lepiter'
\end{code}
\item[\texttt{rect(x, y, width, height)}] \hfill \\
	Draws a rectangle of the given width and height, parallel to the coordinate axes with its top left corner at \ct{(x; y)}. If an optional fifth argument is given, corners are rounded by that many pixels.
\item[\texttt{text(string, x, y)}] \hfill \\
	Renders the given \ct{string} with its baseline starting at \ct{(x; y)}.
\item[\texttt{textSize(size)}] \hfill \\
	Sets the size for rendering text in pixels. Default is 12px.
\item[\texttt{triangle(x1, y1, x2, y2, x3, y3)}] \hfill \\
	Draws a triangle with its vertices at the points \ct{(x1; y1)}, \ct{(x2; y2)} and \ct{(x3; y3)}.
\end{description}

\subsection{Colors}
\begin{description}
\item[\texttt{color(r, g, b)}, \texttt{color(gray)}] \hfill \\
	Generates a color object which can be stored in a variable also be used with \ct{fill(...)}, \ct{stroke(...)} and \ct{background(...)}.
\item[\texttt{fill(r, g, b)}] \hfill \\
	Selects the color to use for filling rendered shapes. The color is given as three values in the range of 0 to 255 (red, green and blue respectively). Default is white (255, 255, 255).
\item[\texttt{fill(gray)}] \hfill \\
	Selects the gray scale value to use for filling rendered shapes. The color is given as a single value in the range of 0 to 255 (black/dark to white/light). Default is white (255).
\item[\texttt{noStroke()}] \hfill \\
	Disables borders for future shapes. Equivalent to \ct{strokeWeight(0)}.
\item[\texttt{stroke(r, g, b)}] \hfill \\
	Selects the color to use for the borders of rendered shapes (see \ct{fill(r, g, b)}). Default is black (0, 0, 0).
\item[\texttt{stroke(gray)}] \hfill \\
	Selects the gray scale value to use for the borders of rendered shapes (see \ct{fill(gray)}). Default is black (0).
\item[\texttt{strokeWeight(weight)}] \hfill \\
	Determines the size of drawn borders in pixels. Default is 0.5px.
\end{description}

\subsection{Transforms}
\begin{description}
\item[\texttt{rotate(angle)}] \hfill \\
	Rotates all future shapes by the given angle (in \ct{radians}!) clockwise around the origin.
\item[\texttt{scale(factor)}] \hfill \\
	Linearly scales all future shapes by the given factor from the origin.
\item[\texttt{translate(x, y)}] \hfill \\
	Moves the origin \ct{(0; 0)} for all future shapes (defaults to the upper left corner).
\end{description}

\section{Events}
\begin{description}
\item[\texttt{def draw():}] \hfill \\
	is called repeatedly (up to \ct{frameRate} times per second) for drawing the output.
\item[\texttt{def mouseClicked():}] \hfill \\
	is called whenever a mouse button has been clicked \emph{and} released.
\item[\texttt{def mouseMoved():}] \hfill \\
	is called whenever the mouse has been moved. Alternatively query \ct{mouseX} and \ct{mouseY} in \ct{draw()}.
\item[\texttt{def mousePressed():}] \hfill \\
	is called whenever a mouse button has been pressed. Alternatively query \ct{mousePressed} in \ct{draw()}.
\item[\texttt{def mouseReleased():}] \hfill \\
	is called whenever a mouse button has been released.
\item[\texttt{def setup():}] \hfill \\
	is called once as the program starts.
\end{description}

\section{Mathematics}
\begin{description}
\item[\texttt{cos(angle)}] \hfill \\
	Returns the cosine value for the given angle (measured in radians).
\item[\texttt{int(value)}] \hfill \\
	Rounds the value to an integer.
\item[\texttt{PI}] \hfill \\
	The value of the mathematical constant p.
\item[\texttt{max(a, b)}] \hfill \\
	Returns the larger of the two values (\ct{max(a)} returns the largest value contained in the list \ct{a}).
\item[\texttt{min(a, b)}] \hfill \\
	Returns the smaller of the two values (\ct{min(a)} returns the smallest value contained in the list \ct{a}).
\item[\texttt{radians(angle)}] \hfill \\
	Converts the given angle (measured in degrees) into radians.
\item[\texttt{random(limit)}] \hfill \\
	Returns a random floating point number between 0 and limit (inclusive).
\item[\texttt{randomSeed(seed)}] \hfill \\
	Reinitializes the random generator with the given \ct{seed} number. Using the same number will result in the exact same sequence of pseudo-randomly generated numbers.
\item[\texttt{sin(angle)}] \hfill \\
	Returns the sine value for the given angle (measured in radians).
\item[\texttt{sqrt(value)}] \hfill \\
	Returns the square root of the given value.
\item[\texttt{tan(angle)}] \hfill \\
	Returns the tangent value for the given angle (measured in radians).
\end{description}

\section{Lists}
Lists are objects which provide their own methods. Note that for the following commands, \ct{list} is a variable referencing a \ct{list}.
\begin{description}
\item[\texttt{len(list)}] \hfill \\
	Returns the number of items in this list.
\item[\texttt{list.append(value)}] \hfill \\
	Appends the value to the end of the list.
(\ct{list + [value]} instead produces a \emph{new} list. \ct{list + otherList} produces a \emph{new} list out of two lists.)
\item[\texttt{list.pop()}] \hfill \\
	Removes the last item from the list and returns the removed argument.
(\ct{list[:-1]} instead produces a \emph{new} list without the last item.)
\item[\texttt{list.reverse()}] \hfill \\
	Reverses this list's items.
(\ct{list[::-1]} instead produces a \emph{new} list with its items reversed.)
\item[\texttt{list.sort()}] \hfill \\
	Sorts this list's items.
(\ct{sorted(list)} instead produces a \emph{new} sorted list.)
\end{description}

\section{Miscellanea}
\begin{description}
\item[\texttt{delay(ms)}] \hfill \\
	Waits \ct{ms} milliseconds before continuing (mainly needed for demonstration purposes).
\item[\texttt{frameRate(fps)}] \hfill \\
	Limits the frame rate of animations to a maximum of \ct{fps} frames per second. Default is 30.
\item[\texttt{height}] \hfill \\
	Contains the canvas height as set by \ct{size()}.
\item[\texttt{millis()}] \hfill \\
	Returns the number of milliseconds that have passed since the program has started.
\item[\texttt{mouseX}, \texttt{mouseY}, \texttt{mousePressed}] \hfill \\
	Contains the \ct{x}- and \ct{y}-coordinates of the mouse cursor and whether the mouse has been pressed at the start of a \ct{draw}-phase (undefined outside of \ct{draw}).
\item[\texttt{print(value)}, \texttt{println(value)}] \hfill \\
	Prints the given value into an output console (mainly for debugging and for graphic-less program).
\item[\texttt{str(value)}] \hfill \\
	Turns the value into a string, e.\,g. for concatenating several values for use with \ct{text(...)}.
\item[\texttt{width}] \hfill \\
	Contains the canvas width as set by \ct{size()}.
\end{description}
