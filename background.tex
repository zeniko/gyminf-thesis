%%MAIN: thesis.tex

%%%%% Background Information %%%%%
\chapter{Technical Background} \label{ch_background}
Background knowledge required for understanding the following chapters.

\section{Processing} \label{sc_processing}
Brief overview over the ``Processing'' programming language (along Reas and Fry \cite{Rea14}) and reasons for using it.

\begin{itemize}
\item Processing is taught using a top down approach (cf. \ref{ssc_top_down})
\item Processing is an imperative language with visual primitives, allowing for quick visual results
\item Developed in the early 2000s at MIT Media Lab, based on then popular Java
\item Since Python has become the prevalent teaching language (\xxx{citation needed}), Processing has been extended with a Python mode
\item This allows a seamless transition to using all of Python
\item The original IDE is still based on the JRE and transpile code to Java
\item Built-in structure for animations, interaction, \emph{etc.}
\item Some basic code examples: ...
\item Quick results possible: Flappy Bird, Pong, ...
\item Own experience since the 2010s, originally using \url{https://software.zeniko.ch/ProcessingIDE.zip}
\item Newer alternatives are p5.js, p5.py, \emph{etc.}
\end{itemize}


\section{Glamorous Toolkit} \label{sc_gt}
Brief introduction into GT for the uninitiated and reasons for using it. \cite{Gir23}

\begin{itemize}
\item GT is a fully programmable environment (similar to Wirth's Oberon)
\item Origins of GT in Smalltalk, Squeak, Pharo
\item Easy to inspect, adapt, extend
\item Developed by feenk (nod to Oscar)
\item Base concepts: <gtView>, <gtExample>, ...
\end{itemize}


\section{Moldable Development}
Referring to Nierstrasz and G�rba \cite{Nie24}.

\begin{itemize}
\item Tool should adaptable to data
\item Live exploration by writing throw-away code
\item Quick refactoring for keeping useful code around
\end{itemize}
