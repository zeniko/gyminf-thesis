%%MAIN: thesis.tex

%%%%% Background Information %%%%%
\chapter{Technical Background} \label{ch_background}
Background knowledge required for understanding the following chapters.

\section{Processing} \label{sc_processing}
Brief overview over the ``Processing'' programming language (along Reas and Fry \cite{Rea14}) and reasons for using it.

\begin{todo}
\item Processing is taught using a top down approach (cf. \ref{ssc_top_down})
\item Processing is an imperative language with visual primitives, allowing for quick visual results
\item Developed in the early 2000s at MIT Media Lab, based on then popular Java
\item Since Python has become the prevalent teaching language (\cite{Cod20}), Processing has been extended with a Python mode
\item This allows a seamless transition to using all of Python
\item The original IDE is still based on the JRE and transpile code to Java
\item Built-in structure for animations, interaction, \emph{etc.}
\item Some basic code examples: ...
\item Quick results possible: Flappy Bird, Pong, ...
\item Own experience since the 2010s, originally using \url{https://software.zeniko.ch/ProcessingIDE.zip}
\item Newer alternatives are p5.js, p5.py, \emph{etc.}
\end{todo}


\section{Glamorous Toolkit} \label{sc_gt}
Brief introduction into GT for the uninitiated and reasons for using it. \cite{Gir23}

\begin{todo}
\item GT is a fully programmable environment (similar to Wirth's Oberon)
\item Origins of GT in Smalltalk, Squeak, Pharo
\item Easy to inspect, adapt, extend
\item Developed by feenk (nod to Oscar)
\item Base concepts: \ct{<gtView>}, \ct{<gtExample>}, ...
\item Currently tied to Smalltalk, with other languages such as Python, JavaScript, Java supported through bridges
\item Development happens mainly under MacOS, thus Windows integration lags behind (visible e.\,g. when using PythonBridge)
\end{todo}


\section{Moldable Development}
Referring to Nierstrasz and G�rba \cite{Nie24}.

\begin{todo}
\item Alternative to inspecting static source code or live runtime objects
\item Tool should be adaptable to data
\item Live exploration by writing throw-away code
\item Compare to web development through Web Developer console
\item Quick refactoring for keeping useful code around
\item Development patterns: Moldable Tool, Moldable Object, Throwaway Analysis Tool, Custom View, ...
\item Allows bottom-up development (cf. \ref{ssc_bottom_up}?) and thus e.\,g. implementing support for new languages
\item Languages implemented within GT are inspectable and moldable as far as they're implemented: parser state, tokens, parse tree, bytecode, \emph{etc.}
\item Custom views are cheap to implement, given knowledge of Smalltalk+GT
\item Composed Narrative: Visualize object relations through side-by-side views -- useful as pedagogic tool
\item Moldable Object: Incremental development of objects with state and prior views always available; forces transparency and clean separation
\item Project Diary: Lepiter notebook page with runnable code and live views; document progress -- useable as learning journal for students (similar to Microsoft OneNote), or as project basis (similar to Jupyter)
\item Was very useful for quick prototyping, reusable code, ... (cf. chapter \ref{ch_pa})
\end{todo}
